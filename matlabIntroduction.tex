% !TeX encoding = UTF-8
% !TeX program = pdfLaTeX
% !TeX root = matlab-exercises-emaip.tex
% !TeX spellcheck = en_GB
\section{Matlab introduction}
\todo[inline]{Add Matlab introduction exercises}
\todo[inline]{Add exercises about systems of linear equations}


\subsection{Creating functions}

\subsection{If statements}

\subsection{For loops}

\subsection{While loops}


\subsection{Plot of functions}

\begin{ex}
Plot the function $f(x) = cos(x) + 3*sin(3*x)$
over the $x$ range $[-2, 2]$.
Use the following template 
\begin{lstlisting}
% Create evenly distributed x values that covers the range
x = linspace(0, 10, 100);

% Calculate the associated y value
y = sin(x);

% Open a figure and plot the associated x and y values
figure(1);
plot(x, y);
\end{lstlisting}
\begin{hint}
\end{hint}
\begin{sol}
A solution is:
\begin{lstlisting}
% Create evenly distributed x values that covers the range
x = linspace(-2, 2, 100);

% Calculate the associated y value
y = cos(x) + 3*sin(3*x);

% Open a figure and plot the associated x and y values
figure(1);
plot(x, y);
\end{lstlisting}
\end{sol}
\end{ex}



\begin{ex}
Plot the function $f(x) = 1 / (1.1 + sin(4*x))$
over the interval $x \in [-3, 3]$.
\begin{hint}
\end{hint}
\begin{sol}
A solution is:
\begin{lstlisting}
% Create evenly distributed x values that covers the range
x = linspace(-3, 3, 100);

% Calculate the associated y value
y = 1 ./ (1.1 + sin(4*x));

% Open a figure and plot the associated x and y values
figure(1);
plot(x, y);
\end{lstlisting}
\end{sol}
\begin{solutionfile}{plot}
% Create evenly distributed x values that covers the range
x = linspace(-3, 3, 100);

% Calculate the associated y value
y = 1 ./ (1.1 + sin(4*x));

% Open a figure and plot the associated x and y values
figure(1);
plot(x, y);
\end{solutionfile}
\end{ex}


\subsection{Plot of secant lines}

\begin{ex}
Draw multiple secant lines on the same figure.
The distance between the points that define the secant line
should be reduced before plotting the next secant line.
Use the following distances between the $x$ values:
$h \in {0.2, 0.1, 0.05, 0.02, 0.01}$.
Please use the template below
\begin{lstlisting}
% Create x values
x = linspace(-3, 6, 100);

% Create a function handle for the function to plot.
fh = @(x) 2*x.^2 - x + 1;

% Find the two end points of the secant to plot.
h = 2;
x0 = 0.2;
x1 = x0 + h;
y0 = fh(x0);
y1 = fh(x1);

% Open a figure and plot the x and y values.
figure(2);
hold off;
% Plot the function
plot(x, fh(x));
hold on;
plot([x0, x1], [y0, y1], 'o-');
\end{lstlisting}
\begin{hint}
\end{hint}
\begin{sol}
A solution is:
\begin{lstlisting}
\end{lstlisting}
\end{sol}
\end{ex}

%
%%% Indskrive vektorer og matricer
%A = [1, 2, 3; 2, 3, 1; 1, 1, 1]
%B = [1, 1, 1; 1, 2, 4; 7, 6, 5]
%C = [1; 2; 3]
%
%
%%% Addition
%disp('addition');
%A + A
%A + B
%B + A
%B + B
%
%
%%% Multiplikation
%disp('multiplikation');
%A * A
%A * B
%B * A
%B * B
%
%
%%% 
%% Transponering
%A
%transpose(A)
%A'
%
%
%%% Opgave
%% Find nogle opgaver med matricer og vektorer.
%% Tjek jeres resultater ved at få Matlab til at udføre de samme beregninger
%% som I har gjort i hånden.
%
%
%
%%%
%% Løsning af lineære ligningssystemer
%
%% Givet ligningerne
%% 2*x1 + 5*x2 = 2 og -4*x1+3*x2 = -30.
%% Kan koefficient matricen A og værdi vektoren b opskrives.
%A = [2, 5; -4, 3];
%b = [2; -30];
%
%% Ud fra værdi vektoren og koefficient matricen, kan 
%% løsningen til ligningssystemet bestemmes på følgende 
%% tre forskellige måder.
%A \ b
%inv(A) * b
%linsolve(A, b)
%
%
%%%
%% Opgave
%% Plot ligningerne x + 2y = 3 og 2x - y = 2 i det samme 
%% koordinatsystem.
%% Opskriv ligningsystemet på matrix form og løs det via 
%% linsolve eller lign.
%% Plot løsningen sammen med de to ligninger.
%% Tjek i hånden at det fremkomne resultat løser de oprindelige 
%% ligninger.
%% teknik.
%
%
%
%
%%%
%% Opgave
%% Der er en fejl i den gauss elimination der fremgår herunder.
%% Benyt funktionen rref (reduced row echelon form), til at 
%% finde fejlen, ved at anvende rref på de forskellige delresultater.
%[1, 2, 3, 1; 3, 2, 1, 1; 1, 1, 2, 1]
%[1, 2, 3, 1; 0, -4, -8, -2; 0, -1, -1, 0]
%[1, 2, 3, 1; 0, 1, -2, 0.5; 0, -2, -1, 0]
%[1, 2, 3, 1; 0, 1, -2, 0.5; 0, 0, -5, 1]

%
%\subsection{Compare derivatives}
%
%\begin{ex}
%\begin{lstlisting}
%%% Sammenlign funktion og dens afledte
%x = linspace(-0.5, 6.2, 100);
%fh = @(x) sin(x);
%fh_prime = @(x) cos(x);
%
%figure(3);
%hold off;
%plot(x, fh(x));
%hold on;
%plot(x, fh_prime(x), 'r');
%legend("f(x)", "f'(x)")
%title("Symbolsk afledte")
%
%
%%% Numerisk estimering af afledte
%% Definer x vaerdier funktionen skal plottes for
%x = linspace(-0.5, 6.2, 100);
%dx = 0.001;
%xprime = x + dx;
%
%% Beregn y vaerdier som skal plottes
%fh = @(x) sin(x);
%y = fh(x);
%yprime = sin(xprime);
%dy = yprime - y;
%derivative = dy / dx;
%
%% Aabn en figur og plot x og y vaerdierne.
%figure(2);
%plot(x, fh(x));
%% Hold fast i det der allerede er plottet.
%hold on;
%plot(x, derivative);
%hold off;
%legend("f(x)", "f'(x)")
%title("Numerisk afledte")
%
%
%
%%% Opgave
%% Sammenlign den analytisk afledte af funktionen 
%% f(x) = (3*x^2+sin(4*x))^2
%% med den numerisk afledte.
%% Benyt gerne koden herover.
%
%
%
%%% Numerisk beregning af graense vaerdier (1.1, 1.01, 1.001, 1.0001, 1.00001, ...)
%
%% Soerg for at matlab skriver komma tal med mange decimaler
%% efter kommaet.
%format long;
%
%% Definer funktionen der skal undersoeges, og hvilken vaerdi x
%% skal gaa mod i graensen.
%fh = @(x) (x.^2 - 1) ./ (x - 1);
%xval = 1;
%
%% Lav en liste med en raekke vaerdier, der gradvist kommer naermere
%% graensevaerdien.
%xvals = xval + 10.^(-linspace(1, 10, 10));
%approx_limits = fh(xvals);
%figure(3);
%plot(xvals, approx_limits, 'o:');
%
%% Skift tilbage til normal visning af kommatal.
%format short;
%
%\end{lstlisting}
%\begin{hint}
%\end{hint}
%\begin{sol}
%A solution is:
%\begin{lstlisting}
%\end{lstlisting}
%\end{sol}
%\end{ex}



\subsection{Find minima}

%%% Finde minima af en funktion
%% Lav en function handle til den funktion der skal plottes.
%% Det gør det lettere at tilpasse koden senere.
%% Notationen med x.^2, handler om at funktionen skal virke på
%% vektorer.
%fh = @(x) 2*x.^2 - x + 1;
%initial_guess = 1;
%x_min = fminsearch(fh, initial_guess);
%
%% Definer x værdier funktionen skal plottes for
%x = linspace(-1, 2, 100);
%
%% Åbn en figur og plot x og y værdierne.
%figure(1);
%hold off;
%plot(x, fh(x));
%hold on;
%plot(x_min, fh(x_min), 'o');
%
%
%%%
%% Opgave
%% Bestem minima af funktionen f(x) = x + 2/x.
%% Se udelukkende på positive værdier af x.
%% Plot funktionen og indsæt et passende startgæt i 
%% kaldet til fminsearch.
%
%
%
%
%%%
%% Opgave
%% Bestem minima af funktionen f(x) = x^2 + 1 / x.
%
%
%
%
%
%
%%% Finde maksimum af en funktion

\begin{lstlisting}
% Make a function handle to the function that is to be
% plotted. This makes it easier to adapt the code later.
% The notation \verb!x.^2!, is to ensure that the function
% also works on vectors.
fh = @(x) -2*x.^2 - 3*x + 1;

% Locate the maximum value of the function (notice the
% minus in the function handle). 
% Use an initial guess of x = 1.
x_min = fminsearch(@(x) -fh(x), 1);

x = linspace(-1, 2, 100);

% Open a figure and plot associated x and y values.
figure(1);
hold off;
plot(x, fh(x));
hold on;
% Indicate the location of the maxima.
plot(x_min, fh(x_min), 'o');
\end{lstlisting}


%
%
%
%%%
%% Opgave
%% Find et lokalt maksimum af funktionen 
%% f(x) = cos(3 - x) * exp(-x^2)
%
%
%%%
%% Opgave
%% Find det globale maksimum af funktionen 
%% f(x) = cos(3 - x) * exp(-x^2)
%



\subsection{Plotting inverse functions}

\begin{ex}
Plot the function $f(x) = \sin(x)$ on the interval $x \in [-\pi/2, \pi/2]$
and determine whether the function has an inverse.
The template below might be helpful.
\begin{lstlisting}
% Plot of functions and their inverse
x = linspace(0, 3, 100);
y = sqrt(x);
figure(1);
hold off;
plot(x, y);
hold on; 
plot(y, x);
plot(x, x, 'k:');
legend('f(x)', 'inverse function');
\end{lstlisting}
\begin{hint}
If a function has an inverse, it should be one-to-one (en-en-tydig).
\end{hint}
\begin{sol}
A solution is:
\begin{lstlisting}
x = linspace(0, 3, 100);
y = sin(x);
figure(1);
hold off;
plot(x, y);
hold on; 
plot(y, x);
plot(x, x, 'k:');
legend('f(x)', 'inverse function');
\end{lstlisting}
\end{sol}
\end{ex}



\begin{ex}
Plot the function $f(x) = \cos(x)$ on the interval $x \in [-pi/2, pi/2]$
and determine whether the function has an inverse.
\begin{lstlisting}
\end{lstlisting}
\begin{hint}
If a function has an inverse, it should be one-to-one (en-en-tydig).
\end{hint}
\begin{sol}
A solution is:
\begin{lstlisting}
\end{lstlisting}
\end{sol}
\end{ex}




\begin{ex}\label{exfzero}%
Solve the equation $0 = 4 + x - x^2$ using the \verb!fzero! function.
How many solutions would you expect to find? How many solutions do 
you actually find?
The template below might be useful:
\begin{lstlisting}
fh = @(x) 0.2 + sin(x);
x = linspace(0, 6, 100);
y = fh(x);
figure(2);
hold off;
plot(x, y); 
guess = 2;
root = fzero(fh, guess)
hold on
plot(root, 0, 'o');
\end{lstlisting}
\begin{hint}
\end{hint}
\begin{sol}
A solution is:
\begin{lstlisting}
\end{lstlisting}
\end{sol}
\end{ex}

\begin{ex}
This exercise is a continuation from \ref{exfzero}.
Get details from \verb!fzero! about how the solution was found.
Look at the documentation to see how.
\begin{hint}
\begin{lstlisting}
help fzero
\end{lstlisting}
\end{hint}
\begin{sol}
A solution is:
\begin{lstlisting}
options = optimset('Display','iter');
fh = @(x) 0.2 + sin(x);
guess = 2;
[x, fval] = fminsearch(fh, guess, options)
\end{lstlisting}
\end{sol}
\end{ex}
 