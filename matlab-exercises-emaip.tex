% !TeX encoding = UTF-8
% !TeX program = pdfLaTeX
% !TeX root = matlab-exercises-emaip.tex
% !TeX spellcheck = en_GB
\documentclass[12pt,a4paper]{article}
\usepackage[colorlinks, linkcolor=blue, citecolor=blue, urlcolor=blue]{hyperref}
%\usepackage[nosolutionfiles]{answers}
\usepackage{answers}
\usepackage{graphicx}
\usepackage{amsmath}
\usepackage{todonotes}
\usepackage{listings}
\usepackage{color}
\definecolor{dkgreen}{rgb}{0,0.6,0}
\definecolor{gray}{rgb}{0.5,0.5,0.5}

\lstset{language=Matlab,
   keywords={break,case,catch,continue,else,elseif,end,for,function,
      global,if,otherwise,persistent,return,switch,try,while},
   basicstyle=\footnotesize\ttfamily,
   keywordstyle=\color{blue},
   commentstyle=\color{dkgreen},
   stringstyle=\color{dkgreen},
%   numbers=left,
%   numberstyle=\tiny\color{gray},
%   stepnumber=3,
%   numbersep=10pt,
   backgroundcolor=\color{white},
   tabsize=4,
   showspaces=false,
   showstringspaces=false}



% Fix for moving the hypertarget one line up.
% https://tex.stackexchange.com/a/412381/1366
\makeatletter
 \newcommand{\linkdest}[1]{\Hy@raisedlink{\hypertarget{#1}{}}}
\makeatother

\newcounter{ex}
\numberwithin{ex}{section}

\newenvironment{ex}[1][]{%
\bigskip
\refstepcounter{ex}
\noindent
\textbf{\linkdest{\theex{}exercise}{}Exercise \theex{}: #1\hfill\hyperlink{\theex{}hint}{hint}, \hyperlink{\theex{}solution}{solution}}\par\noindent}{}

\Newassociation{sol}{Solution}{ans}
\Newassociation{hint}{Hints}{hnt}

\renewcommand{\Hintslabel}[1]{{\linkdest{#1{}hint}{}\textbf{Exercise \hyperlink{#1{}exercise}{#1}:}}\par}
\renewcommand{\Solutionlabel}[1]{{\linkdest{#1{}solution}{}\textbf{Exercise \hyperlink{#1{}exercise}{#1}:}}\par}

\Opensolutionfile{ans}[ans]
\Opensolutionfile{hnt}[hints]

\author{Henrik Skov Midtiby}
\title{Matlab exercises for E-MAIP 2019}
\date{\today}


\begin{document}
\maketitle

\newpage
\tableofcontents

\newpage
\section*{Introduction}

These notes are used in the course \emph{Mathematics and Introduction to programming}
that is taught on University of Southern Denmark during fall 2019.

The source for this document is available on github, 
\url{https://github.com/henrikmidtiby/matlab-notes}.



\section{Setting git up on Windows}

In this section it is explained how to set up git on 
a Windows computer.
This is also described in this \href{https://tekvideo.sdu.dk/t/henrikmidtiby/E-MAIP-2020/2020/1/blok04/4}{video}.

The steps are as follows:
\begin{enumerate}
\item		Download and install \emph{Git for Windows} from \url{https://git-scm.com/}
\item		Open \emph{Git Bash} and identify yourself to git
\item		Generate an ssh key
\item 	Add the generated ssh key to \url{https://gitlab.sdu.dk}
\end{enumerate}

\subsection{Download and installation of \emph{Git for Windows}}

Open \url{https://git-scm.com/} and click on the download button
in the right side of the page.

\subsection{Open Git Bash and identify yourself to git}

Find the \emph{Git Bash} program and start it.
It will look like a black image with some coloured text in the 
top left corner.
What is opened now is a computer console or terminal.
It can be very effective to do certain tasks on the computer through the console.

To set name and email that the git installation uses on the 
computer, the following commands should be executed:
\begin{verbatim}
git config --global user.name "<your name>"
git config --global user.email "<your email>"
\end{verbatim}

For me it looked like this:
\begin{verbatim}
hemi@TEK-CB-HEMI-01 MINGW64 ~
$ git config --global user.name "Henrik Skov Midtiby"

hemi@TEK-CB-HEMI-01 MINGW64 ~
$ git config --global user.email "hemi@mmmi.sdu.dk"
\end{verbatim}

\subsection{Generate an ssh key}

To work efficiently with git on remote servers like \url{https://gitlab.sdu.dk} or \url{https://github.com}, you need to generate 
an ssh key.
In Git Bash it is done with the command.
\begin{verbatim}
ssh-keygen -t rsa
\end{verbatim}
After executing the command, the \emph{ssh-keygen} program
will ask you a few questions and then the ssh key is generated.
It is appropriate to use the default values, so you can just 
press enter to accept the default values.

On my computer the key generation process looked like this:
\begin{verbatim}
hemi@TEK-CB-HEMI-01 MINGW64 ~
$ ssh-keygen -t rsa
Generating public/private rsa key pair.
Enter file in which to save the key (/c/Users/hemi/.ssh/id_rsa):
Enter passphrase (empty for no passphrase):
Enter same passphrase again:
Your identification has been saved in /c/Users/hemi/.ssh/id_rsa
Your public key has been saved in /c/Users/hemi/.ssh/id_rsa.pub
The key fingerprint is:
SHA256:GBdcG4YGzNMGF9bOZ5oK7P+m+IxO1+8hYpT2g3g/+Ok hemi@TEK-CB-HEMI-01
The key's randomart image is:
+---[RSA 3072]----+
|     oo===+      |
|      +o*o.o     |
|      .+.o.      |
|       + .o o    |
|     .. S  =     |
|      o+ +o      |
|     .o.*o= .    |
|     ..Booo= .   |
|     .+o=*Eoo    |
+----[SHA256]-----+
\end{verbatim}

\subsection{Add ssh key to gitlab.sdu.dk}

As you have now generated an ssh key, you need to locate the 
public part of the key and add that to \url{https://gitlab.sdu.dk}.
This can be done with the following command.
\begin{verbatim}
cat <path to public key>
\end{verbatim}

On my computer the key is placed in this location 
\verb!/c/Users/hemi/.ssh/id_rsa.pub!, which was displayed 
in the output from the key generation step.

\begin{verbatim}
hemi@TEK-CB-HEMI-01 MINGW64 ~
$ cat /c/Users/hemi/.ssh/id_rsa.pub
ssh-rsa AAAAB3NzaC1yc2EAAAADAQABAAABgQCz0X6DyHeyiqcPe8Y+zeQ60Y
4F0sxR2MqNlsL52ZrllRd0toAFvPvmrz9/rT13eHoD+xSIQ5kL44GTZqn9fRmC
lG3ttc1v7Qd/KmAmM5eOpr+cH/c/IbfFGQyiRV7lRz5KES8TFUEs8YNLfLwdHU
cEhmn7x4Zmc7aahTQFTH55EqOt6HU5YoF4vMe0LG+WnhNrh/kwiKN1ndqasEC6
wE3D6K2/Z21j8FcFqvR0PWDubUABQdMN5+huQ7/O7SAVQCeZBYoG2fcn9HRu39
IS8kKQ3AmmrDFUitaWhIWZ+msfmTqJCplYXVlGrlvjcbIVPIgCr/4IL6soWE1V
KL1cEVGV2lAktMqP9cU1E6rYMFMtYIIyjSkE5l2YJZ++A+dgayHOqBx5qPXRq2
k2gRAz8auiV0ztOCRtA09IhzvlH0ZHU126ZoxC72BJpMP6+Rp3r8EJ4LlU2IaI
SXnoSAqXtc8HvngXm74TQpneLSDy+nMKZJ1jpL9dqxtWUM3DbhPyzaM= hemi@
TEK-CB-HEMI-01
\end{verbatim}




\section{2019-09-09 Matlab introduction}
\todo[inline]{Add Matlab introduction exercises}
\todo[inline]{Add exercises about systems of linear equations}

%\newpage
\section{2019-09-23 Complex numbers}
\todo[inline]{Add exercises about complex numbers}


%\newpage
\section{2019-10-03 Unit tests}
\todo[inline]{Add exercises about unit tests}

\newpage
\section{2019-10-10 Fitting models to data}

Here is a set of exercises related to the lesson
on loading data into Matlab and fitting models to data.

\begin{ex}
Make the figure inserted below: \par
\noindent
\includegraphics[width=8cm]{pic/basic_plotting_010.png}
\begin{hint}
The commands \verb!plot!, \verb!xlim! and \verb!ylim! can be useful.
\end{hint}
\begin{sol}
A solution is:
\begin{verbatim}
figure(1);
clf;
x = [1, 3, 2];
y = [1, 2, 3];
plot(x, y, '-o');
xlim([0, 5]);
ylim([0, 5]);
\end{verbatim}
\end{sol}
\end{ex}

\begin{ex}
Find a root of the function $f(x) = e^{-x} - x$.
\begin{hint}
The command \verb!fzero! can be useful.
\end{hint}
\begin{sol}
A solution is:
\begin{verbatim}
fh = @(x) exp(-x) - x
root1 = fzero(fh, 1)
\end{verbatim}
\end{sol}
\end{ex}

\begin{ex}
Find the minimum of the function $f(x) = e^x - 2x$.
\begin{hint}
The command \verb!fminsearch! can be useful.
\end{hint}
\begin{sol}
A solution is:
\begin{verbatim}
fh = @(x) -2x + exp(x);
fminsearch(fh, 1)
\end{verbatim}
\end{sol}
\end{ex}


\begin{ex}
Plot the functions $f(x) = e^x$ and $g(x) = 4x$ over the interval $x \in [-1, 3]$.
\begin{hint}
Define two function handles, one to each function.
Use the functions \verb!linspace!, \verb!hold on!, \verb!plot! and \verb!figure!.
\end{hint}
\begin{sol}
A solution is:
\begin{verbatim}
fh = @(x) exp(x);
gh = @(x) 4*x;
x = linspace(-1, 3);
figure(1);
clf;
hold on;
plot(x, fh(x));
plot(x, gh(x))
\end{verbatim}
\end{sol}
\end{ex}


\begin{ex}
Make the figure inserted below. Pay attention to the axes labels and the width of the plotted line. The visualized function is $f(x) = x^2 - x$. \par
\noindent
\includegraphics[width=8cm]{pic/basic_plotting_020.png}
\begin{hint}
The commands \verb!plot!, \verb!xlabel! and \verb!ylabel! can be useful.
\end{hint}
\begin{sol}
A solution is:
\begin{verbatim}
figure(1);
clf;
fh = @(x) -x + x.^2;
x = linspace(-3, 3);
plot(x, fh(x), 'LineWidth', 3);
xlabel('x values');
ylabel('y values');
\end{verbatim}
\end{sol}
\end{ex}


\begin{ex}
Find two numerical solutions to the equation
\[
e^{x} = 4x
\]
\begin{hint}
Write the equation on the form $f(x) = 0$ (collect all elements on the left hand side).
The command \verb!fzero! can be useful.
\end{hint}
\begin{sol}
A solution is:
\begin{verbatim}
fh = @(x) exp(x) - 4*x;
root1 = fzero(fh, 1)
root2 = fzero(fh, 3)
\end{verbatim}
\end{sol}
\end{ex}



\begin{ex}
Solve the set of linear equations
\begin{align*}
1	& = 5x + 2y	\\
0	& = -x + y + 3z + 2v	\\
-10 	& = x + 3y - 4z + v	\\
0	& = -2x + 3z - 2v 
\end{align*}
\begin{hint}
Write the system of equations on matrix form $A \cdot \vec{x} = \vec{b}$.
\end{hint}
\begin{sol}
A solution is:
\begin{verbatim}
A = [5, 2, 0, 0; -1, 1, 3, 2; 1, 3, -4, 1; -2, 0, 3, -2];
b = [1; 0; -10; 0];
x = linsolve(A, b)
A * x
\end{verbatim}
\end{sol}
\end{ex}



\begin{ex}
Enter\label{exPlotDataToFitLinearModelTo}
the following values for $x$ and $y$ and plot the points.
Then describe the trend you are observing in the data.
\begin{verbatim}
x = [1, 2, 3, 5];
y = [4, 3, 2, 1];
\end{verbatim}
\begin{hint}
Use the \verb!plot! method.
\end{hint}
\begin{sol}
A solution is:
\begin{verbatim}
x = [1, 2, 3, 5];
y = [4, 3, 2, 1];
figure(1);
clf;
plot(x, y, 'o');
\end{verbatim}
From the plot I see that the $y$ values decreases as the $x$ values increases.
\end{sol}
\end{ex}

\begin{ex}
This\label{exPlotDataToFitLinearModelTo2}
exercise is a continuation of exercise \ref{exPlotDataToFitLinearModelTo}.
Implement a linear model $y = a \cdot x + b$ in matlab. 
The linear model should be defined like below, where P is a vector containing the 
model parameters $a$ and $b$.
\begin{verbatim}
model = @(x, P) <fill in stuff here>;
\end{verbatim}
\begin{hint}
The output of the model should match the output below.
\begin{verbatim}
>> model([1, 2, 3, 5], [-1, 5])
ans =
     4     3     2     0
\end{verbatim}
\end{hint}
\begin{sol}
A solution is:
\begin{verbatim}
model = @(x, P) P(1) * x + P(2);
\end{verbatim}
\end{sol}
\end{ex}

\begin{ex}
This \label{exPlotDataToFitLinearModelTo3}
exercise is a continuation of exercise \ref{exPlotDataToFitLinearModelTo2}.
Implement a method that calculates the squared error
between the model and observations.
\[
\text{squared error} = \sum_i \left(y_i - \text{model}(x_i)\right)^2
\]
Assume that the $x$ and $y$ values are saved in the variables 
\verb!x! and \verb!y! respectively.
\begin{verbatim}
model_error = @(P) <fill in stuff here>;
\end{verbatim}
\begin{hint}
The \verb!sum! function is helpful.
\end{hint}
\begin{sol}
A solution is:
\begin{verbatim}
>> x = [1, 2, 3, 5];
>> y = [4, 3, 2, 1];
>> model = @(x, P) P(1) * x + P(2);
>> model_error = @(P) sum((y - model(x, P)).^2);
>> squared_error = model_error([-1.2, 5])
squared_error =
    4.5600
\end{verbatim}
\end{sol}
\end{ex}

\begin{ex}
This exercise is a continuation of exercise \ref{exPlotDataToFitLinearModelTo3}.
Minimize the squared error, defined in exercise \ref{exPlotDataToFitLinearModelTo3} and then plot the model with the 
found parameter values along with the original data.
\begin{hint}
The \verb!fminsearch! function is helpful, it should converge to 
the proper solution independent of the initial guess.
The solution contains the following values $-0.7429$ and $4.5429$.
\end{hint}
\begin{sol}
A solution is:
\begin{verbatim}
x = [1, 2, 3, 5];
y = [4, 3, 2, 1];
model = @(x, P) P(1) * x + P(2);
model_error = @(P) sum((y - model(x, P)).^2);
P = fminsearch(model_error, [4, 2])

xvals = linspace(0, 6);
figure(1);
clf;
hold on;
plot(x, y, 'o');
plot(xvals, model(xvals, P));
\end{verbatim}
\end{sol}
\end{ex}



\begin{ex}
Plot the three functions given below in the same plot.
\[
f(x) = \frac{1}{x + 1} \qquad \qquad g(x) = \frac{1}{x + 3} \qquad \qquad h(x) = \frac{1}{(x + 1)(x + 3)}
\]
Do the functions have something in common?
\begin{hint}
Look at the discontinuities of the functions. Where are they placed?
\end{hint}
\begin{sol}
A solution is:
\begin{verbatim}
fh = @(x) 1 ./ (x + 1);
gh = @(x) 1 ./ (x + 3);
hh = @(x) 1 ./ ((x + 1) .* (x + 3));
x = linspace(-5, 1, 1000);
figure(1);
clf; 
hold on;
plot(x, fh(x));
plot(x, gh(x));
plot(x, hh(x));
ylim([-4, 4]);
\end{verbatim}
\end{sol}
\end{ex}

\begin{ex}
Given the three functions below:
\[
f(x) = \frac{1}{x + 1} \qquad \qquad g(x) = \frac{1}{x + 3} \qquad \qquad h(x) = \frac{1}{(x + 1)(x + 3)}
\]
Plot $h(x)$ and a linear combination of $f(x)$ and $g(x)$ in the same plot.
Adjust the coefficients of $f(x)$ and $g(x)$, so that the two plotted functions 
becomes as similar as possible.
\begin{hint}
An example of a linear combination of $f(x)$ and $g(x)$ is
\[
2 \cdot f(x) + 3 \cdot g(x)
\]
\end{hint}
\begin{sol}
A solution is:
\begin{verbatim}
fh = @(x) 1 ./ (x + 1);
gh = @(x) 1 ./ (x + 3);
hh = @(x) 1 ./ ((x + 1) .* (x + 3));
figure(2);
clf; 
hold on;
plot(x, 0.5*fh(x) - 0.5*gh(x));
plot(x, hh(x), 'LineStyle', '- -', 'LineWidth', 3);
ylim([-4, 4]);
\end{verbatim}
\end{sol}
\end{ex}




\section{2019-10-24 Numerical integration}

\todo[inline]{Add exercises where some integration rules are tested using numerical methods (this should equal that, check using the integral method in Matlab).}


\begin{ex}[Reference value]%
Evaluate the integral below numerically using the \verb!integral! function.
\[
\int_0^5 e^{-x} \cdot \sin(x) dx
\]
\begin{hint}
The two first significant digits are 0.50.
\end{hint}
\begin{sol}
A solution is:
\begin{verbatim}
fh = @(x) exp(-x) .* sin(x);
integral(fh, 0, 5)
\end{verbatim}
\end{sol}
\end{ex}

\begin{ex}[Trapez rule]%
Evaluate \label{exTrapezRule}
the function:
\[
  e^{-x} \cdot \sin(x) dx
\]
at 10 evenly spread x-values from 0 to 5, the values in this list
will be called $x_i$ in the following.
Then calculate the sum
\[
\sum_{i = 1}^{9} \frac{x_i + x_{i + 1}}{2} \cdot (x_{i + 1} - x_i)
\]
\begin{hint}
Use \verb!linspace! to generate a list of x-values.
\end{hint}
\begin{sol}
A solution is:
\begin{lstlisting}
fh = @(x) exp(-x) .* sin(x);
nvals = 10;
x = linspace(0, 5, nvals);
fx = fh(x);
value = 0;
for k = 1:(nvals - 1)
    avg_height = 0.5*(fx(k) + fx(k + 1));
    dx = x(k + 1) - x(k);
    value = value + avg_height * dx;
end
\end{lstlisting}
\end{sol}
\end{ex}

\begin{ex}[Trapez rule as a function]%
Create a function that uses the Trapez rule to estimate the 
value of a definite integral.
The function must have the signature
\begin{lstlisting}
function res = trapez_rule(fh, x_low, x_high, n_intervals)
\end{lstlisting}
Use the following examples to test the function
\begin{lstlisting}
>> trapez_rule(fh, 0, 5, 10)
ans = 0.4770
>> trapez_rule(fh, 0, 5, 20)
ans = 0.4966
>> trapez_rule(fh, 0, 5, 100)
ans = 0.5021
\end{lstlisting}
\begin{hint}
Use your solution from exercise \ref{exTrapezRule} and build a function around it.
\end{hint}
\begin{sol}
A solution is:
\begin{lstlisting}
function res = trapez_rule(fh, x_low, x_high, n_intervals)

x = linspace(x_low, x_high, n_intervals);
fx = fh(x);
res = 0;
for k = 1:(n_intervals - 1)
    avg_height = 0.5*(fx(k) + fx(k + 1));
    dx = x(k + 1) - x(k);
    res = res + avg_height * dx;
end

end
\end{lstlisting}
\end{sol}
\end{ex}

\begin{ex}
Rewrite the following function such that it uses a for loop.

\begin{lstlisting}
function functionThatRepeatsStuff()
disp(1);
disp(1);
disp(1);
disp(1);
disp(1);
disp(1);
disp(1);
disp(1);
end
\end{lstlisting}
\begin{hint}
Identify what is repeated and how many times it is repeated.
\end{hint}
\begin{sol}
A solution is:
\begin{lstlisting}
function functionThatRepeatsStuff()
for k = 1:8
  disp(1);
end
end
\end{lstlisting}
\end{sol}
\end{ex}



\begin{ex}
Rewrite the following function such that it uses a for loop.
\begin{lstlisting}
function functionThatCounts()
disp(1);
disp(2);
disp(3);
disp(4);
disp(5);
disp(6);
disp(7);
disp(8);
disp(9);
end
\end{lstlisting}
\begin{hint}
Use a for loop, with a structure as shown here:
\begin{lstlisting}
for k = 1:4
end
\end{lstlisting}
\end{hint}
\begin{sol}
A solution is:
\begin{lstlisting}
function functionThatCounts()
for k = 1:9
  disp(k);
end
end
\end{lstlisting}
\end{sol}
\end{ex}



\begin{ex}
Create \label{exSumOfPositiveIntegersLessThanN}
a function that takes one positive integer, n, as argument and
calculates the sum of all integers in the range $1, 2, ..., n$.

\begin{lstlisting}
function res = sumOfIntegersLessThanOrEqualTo(n)
\end{lstlisting}
Example usage of the function:
\begin{lstlisting}
>> sumOfIntegersLessThanOrEqualTo(1);
>> sumOfIntegersLessThanOrEqualTo(1)
ans = 1
>> sumOfIntegersLessThanOrEqualTo(5)
ans = 15
>> sumOfIntegersLessThanOrEqualTo(10)
ans = 55
>> sumOfIntegersLessThanOrEqualTo(100)
ans = 5050
\end{lstlisting}
\begin{hint}
Use a for loop, with a structure as shown here:
\begin{lstlisting}
for k = 1:n
end
\end{lstlisting}
Prior to the for loop, create a variable and set its value to 0.
Then update the value in the variable for each cycle in the for loop.
\end{hint}
\begin{sol}
A solution is:
\begin{lstlisting}
function res = sumOfIntegersLessThanOrEqualTo(n)
res = 0;
for k = 1:n
  res = res + 1;
end
end
\end{lstlisting}
\end{sol}
\end{ex}




\begin{ex}
Create a function that takes one positive integer, n, as argument and
calculates the sum of all integers in the range $1, 2, ..., n$, which are even.

\begin{lstlisting}
function res = sumOfEvenIntegersLessThanOrEqualTo(n)
\end{lstlisting}
Example usage of the function:
\begin{lstlisting}
>> sumOfEvenIntegersLessThanOrEqualTo(1);
>> sumOfEvenIntegersLessThanOrEqualTo(1)
ans = 0
>> sumOfEvenIntegersLessThanOrEqualTo(5)
ans = 6
>> sumOfEvenIntegersLessThanOrEqualTo(10)
ans = 30
>> sumOfEvenIntegersLessThanOrEqualTo(100)
ans = 2550
\end{lstlisting}
\begin{hint}
Modify the solution to exercise \ref{exSumOfPositiveIntegersLessThanN}
so it only increments the variable when $k$ is even.
\end{hint}
\begin{sol}
A solution is:
\begin{lstlisting}
function res = sumOfEvenIntegersLessThanOrEqualTo(n)
res = 0;
for k = 1:n
  if(mod(k, 2) == 0)
    res = res + 1;
  end
end
end
\end{lstlisting}
\end{sol}
\end{ex}



\begin{ex}
Create a function that calculates the n'th Fibonacci number. The nth
fibonacci number can be calculated using the formula: 
\[
F_n = F_{n-1} + F_{n-2}
\]
with the two base cases $F_0 = 0$ and $F_1 = 1$. The sequence goes like
$0, 1, 1, 2, 3, 5, 8, 13, 21, 34, ...$

\begin{lstlisting}
function res = fib(n)
\end{lstlisting}
Example usage of the function:
\begin{lstlisting}
>> fib(0);
>> fib(0)
ans = 0
>> fib(1)
ans = 1
>> fib(6)
ans = 8
>> fib(17)
ans = 1597
>> fib(29)
ans = 514229
\end{lstlisting}
\begin{hint}
Use one or more if statements to ensure that the two base
cases are handled properly.
Then use the relation
\[
\textrm{fib}(n) = \textrm{fib}(n - 1) + \textrm{fib}(n - 2)
\]
\end{hint}
\begin{sol}
A solution is:
\begin{lstlisting}
function res = fib(n)

if(n < 2)
  res = n;
else
  res = fib(n - 1) + fib(n - 2);
end
end
\end{lstlisting}
\end{sol}
\end{ex}






\section{2019-11-04 Numerical solutions of differential equations}


\subsection{Euler's method}


\begin{ex}[Eulers method]%
Function definition:
\begin{lstlisting}
function [yvals, fcalls] = euler(fnc, xvals, y0)
% Uses the Euler method for approximating the first order
% differential equation defined by the function handle fnc.
%
% Input values:
% - fnc Function handle to a function which takes two
% input arguments and returns a scalar value.
% Eg. @(x, y) (y+sin(x))
% - xvals List of x values where the corresponding y
% values should be calculated.
% - y0 Initial state of the dependent function.
%
% Output values:
% - yvals Approximation of y(x) at the locations
% specified in xvals.
% - fcalls Number of function evaluations.
\end{lstlisting}
Example usage of the function:
\begin{lstlisting}
>> [yvals, fevals] = euler(@(x, y) y, [0:5], 1)
yvals =
1 2 4 8 16 32
fevals =
5
>> euler(@(x, y) y, [0:5], 2)
ans =
2 4 8 16 32 64
>> euler(@(x, y) 0.1*y, [0:5], 1)
ans =
1.0000 1.1000 1.2100 1.3310 1.4641 1.6105
>> euler(@(x, y) 0.1*y+0.2*x, [0:5], 1)
ans =
1.0000 1.1000 1.4100 1.9510 2.7461 3.8207
\end{lstlisting}
\begin{hint}
\end{hint}
\begin{sol}
A solution is:
\begin{lstlisting}
\end{lstlisting}
\end{sol}
\end{ex}

\begin{ex}
\label{excEulerConvergence} \\
For the initial value problem
\begin{align}
y'(t)
	& = 1.2 \cdot y(t) 	&
y(0) = 1
\end{align}
Determine the analytic solution and calculate the exact value of $y(3)$.
Use Euler's method to approximate $y(3)$ with different step lengths $h$.
How is the error related to the used step length?
\begin{hint}
\end{hint}
\begin{sol}
A solution is:
\begin{verbatim}
\end{verbatim}
\end{sol}
\end{ex}



\Closesolutionfile{hnt}
\Closesolutionfile{ans}


\newpage
\section{Hints}
\input{hints}


\newpage
\section{Solutions}
\input{ans}


\newpage
\section{Links to other resources}

Beginning Matlab Exercises, by R. J. Braun: 
\url{http://www.math.udel.edu/~braun/M349/Matlab_probs2.pdf}

\url{http://kom.aau.dk/~borre/matlab7/exercise.pdf}


\end{document}
