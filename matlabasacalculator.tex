% !TeX encoding = UTF-8
% !TeX program = pdfLaTeX
% !TeX root = matlab-exercises-emaip.tex
% !TeX spellcheck = en_GB
\section{Matlab as a calculator}

\newcommand{\known}[1]{\todo[inline, color=green!20, nolist]{New commands: \texttt{#1}}}

The simplest way of using Matlab, is as a basic pocket calculator, where an
expression is entered and Matlab then calculates the value of the expression.
Such an interaction looks like:
\begin{verbatim}
>> 1+2*3
ans =
     7
\end{verbatim}
Multiple things are happening here, so lets take a closer look at each of the
elements.
The prompt \verb!>>! is the first thing to recognize.
This prompt tells the user, that Matlab is waiting for some input to work on.
In this case the user then entered the expression \verb!1+2*3! and pressed 
\verb!enter!.
The response from Matlab is divided over two lines and contains the following 
\verb!ans = 7!.
When Matlab calculates a value, the value is stored in a variable.
If no variable name is specified, the default variable name \verb!ans! is used, 
this is the case here.
So Matlab tells us that the value of the entered expression has the value 7.



\todo[inline]{Write a about how to use Matlab as a pocket calculator.}


\begin{ex}
For each line of code below, state what value will be saved in each variable.
\begin{verbatim}
a = 1;             % a = 
b = 1;             % a =    b = 
c = a + b;         % a =    b =    c = 
a = b;             % a =    b =    c = 
b = c;             % a =    b =    c = 
c = a + b;         % a =    b =    c = 
a = b;             % a =    b =    c = 
b = c;             % a =    b =    c = 
c = a + b;         % a =    b =    c = 
\end{verbatim}
\begin{sol}
\begin{verbatim}
a = 1;             % a = 1
b = 1;             % a = 1  b = 1 
c = a + b;         % a = 1  b = 1  c = 2
a = b;             % a = 1  b = 1  c = 2
b = c;             % a = 1  b = 2  c = 2
c = a + b;         % a = 1  b = 2  c = 3
a = b;             % a = 2  b = 2  c = 3
b = c;             % a = 2  b = 3  c = 3
c = a + b;         % a = 2  b = 3  c = 5
\end{verbatim}
\end{sol}
\end{ex}



\begin{ex}
Calculate $5 + 3 \cdot 10$\\
Calculate the sum of 5 + 3 and multiply this sum with 10\\
Calculate 10 - 16/4\\
Calculate the difference 10 - 16 and divide this difference with 4
\begin{hint}
Just write it as you would on a normal calculator, eg. a TI Inspire.
\begin{verbatim}
5 + 3 * 10
\end{verbatim}
\end{hint}
\end{ex}
\known{+, -, *, /}

\begin{ex}
Assign the value 2 to the variable $x0$ and hide the answer from this intermediate
calculation.\\
Assign the variable $x1$ the value of $3 \cdot x0^2$ + 4 and divide with 4. Check if the answer to
this calculation is correct (you do not need a pocket calculator to do this).
\end{ex}
\known{=, \^}


\begin{ex}
Solve the second order polynomial
\begin{align}
A x^2 + B x + C = 0
\end{align}
using the formula
\begin{align}
D & = B^2 - 4 \cdot A \cdot C	\\
X & = \frac{-B\pm \sqrt{D}}{2 \cdot A}
\end{align}
Use the following values for A, B and C.\\ \noindent
(A, B, C) = (1, 2, 4), (1, 4, -1), (4, -5, 3), (25, 12, 0), (34, 0, 0) and (0, 2, 0).
\end{ex}
\known{sqrt}

\begin{ex}
Get the value of e and $\pi$ from Matlab.
\end{ex}
\known{exp, pi}

\begin{ex}
Calculate $\sin(3/4\pi)$, $\cos(3/4 \cdot \pi)$, $\tan(3/4 \cdot \pi)$ and $\cot(3/4 \cdot \pi)$.
\end{ex}
\known{sin, cos, tan, cot}

\begin{ex}
Define and print to screen the parameters x and y, where x has to have the values 0, 1,
2, 3, 4, 5, 6, 7, 8, 9, 10 and y has to have the values 0, 0.5, 1, 1.5, 2, 2.5, 3, 3.5, 4, 4.5,
5.\\
Create a Cartesian plot of y against x.
\end{ex}
\known{plot}


\begin{ex}
Define the parameter $x$, where $x$ have to contain $N$ values equally spaced between $0$
and $1$. Use either the built-in function \texttt{linspace} or colon notation.\\
Define a variable $y$ equal to $\cos(4\cdot \pi \cdot x)$.\\
First use $N = 5$ and create a Cartesian plot of $y$ against $x$.\\
Then increase $N$ until you get a smooth plot of the function $y$.
\end{ex}
\known{linspace, :}

\begin{ex}
Define the two row vectors $\vec{v}$ and $\vec{u}$:
\begin{align*}
\vec{v} = (1, 2, 3) \qquad \vec{u} = (3, 4, 1)
\end{align*}
and determine their 
sum ($\vec{v} + \vec{u}$),
difference ($\vec{v} - \vec{u}$),
dot product ($\vec{v} \cdot \vec{u}$) and
cross product ($\vec{v} \times \vec{u}$).
In addition compute the elementwise product, division, power of 2 and 
the transposed and length of the two vectors.
\end{ex}
\known{[\ldots], .*, ./, .\^{}2, .', transpose}


\begin{ex}
Define the two matrices
\begin{align*}
\textrm{AMAT} = \begin{pmatrix}
1 &2 &3	\\
2 &4 &6	\\
1 &2 &5
\end{pmatrix}
\qquad
\textrm{BMAT} =
\begin{pmatrix}
1 &0 &0	\\
2 &4 &6	\\
2 &2 &0
\end{pmatrix}
\end{align*}
And calculate their product AMAT $\cdot$ BMAT, element wise product of AMAT times BMAT, sum
AMAT+BMAT, transposed, inverted, determinants, sizes, column and row numbers.
\end{ex}
\known{[,;], \^{}-1, inv, det, size}


\begin{ex}
Use the matrices from the previous exercise. What is the value of \verb!AMAT(5)! and why?
\end{ex}


\begin{ex}
Construct an Identity Matrix $I_{5\times5}$ and a diagonal matrix
\begin{align*}
\textrm{CMAT} = \begin{pmatrix}
2 &0 &0 &0 &0	\\
0 &4 &0 &0 &0	\\
0 &0 &7 &0 &0	\\
0 &0 &0 &0.5 &0	\\
0 &0 &0 &0 &1	
\end{pmatrix}
\end{align*}
And calculate their product $I_{5 \times 5} \cdot \textrm{CMAT}$.
\end{ex}
\known{eye, diag}


\begin{ex}
Solve the set of linear equations
\begin{align*}
1	& = 5x + 2y	\\
0	& = -x + y + 3z + 2v	\\
-10 	& = x + 3y - 4z + v	\\
0	& = -2x + 3z - 2v 
\end{align*}
\end{ex}
\known{$\backslash$}




\subsection{Plot of functions}

\begin{ex}
Plot the function $f(x) = \cos(x) + 3 \cdot \sin(3 \cdot x)$
over the $x$ range $[-2, 2]$.
Use the following template 
\begin{lstlisting}
% Create evenly distributed x values that covers the range
x = linspace(0, 10, 100);

% Calculate the associated y value
y = sin(x);

% Open a figure and plot the associated x and y values
figure(1);
plot(x, y);
\end{lstlisting}
\begin{sol}
A solution is:
\begin{lstlisting}
% Create evenly distributed x values that covers the range
x = linspace(-2, 2, 100);

% Calculate the associated y value
y = cos(x) + 3*sin(3*x);

% Open a figure and plot the associated x and y values
figure(1);
plot(x, y);
\end{lstlisting}
\end{sol}
\end{ex}



\begin{ex}
Plot the function $f(x) = 1 / (1.1 + \sin(4 \cdot x))$
over the interval $x \in [-3, 3]$.
\begin{sol}
A solution is:
\begin{lstlisting}
% Create evenly distributed x values that covers the range
x = linspace(-3, 3, 100);

% Calculate the associated y value
y = 1 ./ (1.1 + sin(4*x));

% Open a figure and plot the associated x and y values
figure(1);
plot(x, y);
\end{lstlisting}
\end{sol}
\begin{solutionfile}{plot}
% Create evenly distributed x values that covers the range
x = linspace(-3, 3, 100);

% Calculate the associated y value
y = 1 ./ (1.1 + sin(4*x));

% Open a figure and plot the associated x and y values
figure(1);
plot(x, y);
\end{solutionfile}
\end{ex}
