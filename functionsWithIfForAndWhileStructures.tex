% !TeX encoding = UTF-8
% !TeX program = pdfLaTeX
% !TeX root = matlab-exercises-emaip.tex
% !TeX spellcheck = en_GB
\section{Functions with if, for and while structures}

\begin{ex}
Create a function which takes a list and a pivot element 
as input. 
Elements in the list that are smaller than the pivot element
should be put in \verb!list1! and elements larger than or equal
should be put in \verb!list2!.
The function signature should be:
\begin{lstlisting}
function [list1, list2] = divide_by_pivot(list, pivot)
\end{lstlisting}
Use the following examples to test the function:
\begin{lstlisting}
>> [l1, l2] = divide_by_pivot([2, 6, 4], 3)
l1 = 2
l2 = 6     4
>> [l1, l2] = divide_by_pivot([2, 6, 4, 1, 4], 7)
l1 = 2     6     4     1     4
l2 = []
>> [l1, l2] = divide_by_pivot([2, 6, 4, 1, 4], 3)
l1 = 2     1
l2 = 6     4     4
\end{lstlisting}
\begin{hint}
Use a for loop to iterate over all elements in the input 
list.
Use an if statement to select which list the element should 
be added to.
\end{hint}
\begin{sol}
A solution is:
\begin{lstlisting}
function [list1, list2] = divide_by_pivot(list, pivot)

list1 = [];
list2 = [];
for val = list
    if val < pivot
        list1 = [list1, val];
    else
        list2 = [list2, val];
    end
end

end
\end{lstlisting}
\end{sol}
\end{ex}

\begin{ex}
Create a function that takes a list as input.
All elements in the list larger than zero should 
be put in a list that is then returned.

The function signature should be:
\begin{lstlisting}
function list = keep_positive_elements(input_list)
\end{lstlisting}
Use the following examples to test the function:
\begin{lstlisting}
>> keep_positive_elements([1, -2, 3, -4, 5])
ans = 1     3     5
>> keep_positive_elements([-1, -2, -3, -4, 5])
ans = 5
>> keep_positive_elements([6, 4, 2, 1])
ans = 6     4     2     1
\end{lstlisting}
\begin{hint}
Use a for loop to iterate over all values in 
the input list.
Use an if statement to decide whether the value
should be added to the output.
\end{hint}
\begin{sol}
A solution is:
\begin{lstlisting}
function list = keep_positive_elements(input_list)

list = [];
for val = input_list
    if val > 0
        list = [list, val];
    end
end

end
\end{lstlisting}
\end{sol}
\end{ex}


\begin{ex}
Create a function that takes a list of $n$ points as input 
(in the form of a $n$ by $2$ matrix).
The function should return a matrix with distances between 
the input points, so the element $a_{12}$ in the matrix 
contains the distance between the first and the second point
in the input list.
The function signature should be:
\begin{lstlisting}
function res = distance_matrix(list_of_points)
\end{lstlisting}
Use the following examples to test the function:
\begin{lstlisting}
>> distance_matrix([1, 0; 1, 1])
ans =
     0     1
     1     0
>> distance_matrix([1, 0; 0, 1; 1, 1])
ans =
         0    1.4142    1.0000
    1.4142         0    1.0000
    1.0000    1.0000         0
>> distance_matrix([1, 0; 1, 1; 4, 4])
ans =
         0    1.0000    5.0000
    1.0000         0    4.2426
    5.0000    4.2426         0
\end{lstlisting}
\begin{hint}
To calculate the distance matrix, all combinations
of two points should be examined.
This can be achieved by using two for loops placed 
inside each other (so you have an outer loop and an 
inner loop).
\end{hint}
\begin{sol}
A solution is:
\begin{lstlisting}
function res = distance_matrix(list_of_points)

[s1, s2] = size(list_of_points);
res = zeros(s1, s1);

for idx1 = 1:s1
    for idx2 = 1:s1
        p1 = list_of_points(idx1, :);
        p2 = list_of_points(idx2, :);
        difference = p1 - p2;
        distance = norm(difference);
        res(idx1, idx2) = distance;
    end
end

end
\end{lstlisting}
\end{sol}
\end{ex}


\begin{ex}
Create a function which takes two lists as input.
The output should be the two lists combined, 
so that all elements in the first list appears
before all elements in the second list.

The function signature should be:
\begin{lstlisting}
function res = concatenate_lists(list1, list2)
\end{lstlisting}
Use the following examples to test the function:
\begin{lstlisting}
>> concatenate_lists([1, 2, 3], [4, 5])
ans =
     1     2     3     4     5
>> concatenate_lists([5, 4], [1, 2, 3])
ans =
     5     4     1     2     3
\end{lstlisting}
\begin{hint}
Use a while loop to move all elements from list1 
to the output list.
Then use a new while loop to move all elements from 
list2 to the output list.
\end{hint}
\begin{sol}
A solution is:
\begin{lstlisting}
function res = concatenate_lists(list1, list2)

res = [];
idx = 1;
while idx <= length(list1)
    res = [res, list1(idx)];
    idx = idx + 1;
end

idx = 1;
while idx <= length(list2)
    res = [res, list2(idx)];
    idx = idx + 1;
end

end
\end{lstlisting}
\end{sol}
\end{ex}


\begin{ex}
\label{exSecantMethodOneStep}%
Implement a function that takes a function handle, and two 
$x$ values.
Use one step of the secant method to make an improved guess of the
root of a function, by using the two $x$ values as guesses.
The calculation to be done in the function is the following
where $f(x)$ is the function specified by the function handle
applied to the value $x$, and $x_1$ and $x_2$ are the two
guesses.
\[
x=x_{2}-f(x_{2}) \cdot {\frac {x_{2}-x_{1}}{f(x_{2})-f(x_{1})}}.
\]
You can read more about the secant method on \href{https://en.wikipedia.org/wiki/Secant_method}{wikipedia}.

The function signature should be:
\begin{lstlisting}
function [x_improved, fval] = sekant_one_step(fh, x1, x2)
\end{lstlisting}
Use the following examples to test the function:
\begin{lstlisting}
>> [x, v] = sekant_one_step(@cos, 1, 2)
x = 1.5649
v = 0.0059
>> [x, v] = sekant_one_step(@cos, 2, 1.5649)
x = 1.5710
v = -1.8238e-04
>> [x, v] = sekant_one_step(@tan, 3, 3.5)
x = 3.1378
v = -0.0038
>> [x, v] = sekant_one_step(@tan, 3.5, 3.1478)
x = 3.1419
v = 2.7253e-04
\end{lstlisting}
\begin{hint}
No if, for or while loop is needed in this exercise.
\end{hint}
\begin{sol}
A solution is:
\begin{lstlisting}
function [x_improved, fval] = sekant_one_step(fh, x1, x2)

fx1 = fh(x1);
fx2 = fh(x2);

slope = (fx2 - fx1) / (x2 - x1);

x_improved = x2 - fx2 / slope;
fval = fh(x_improved);

end
\end{lstlisting}
\end{sol}
\end{ex}


\begin{ex}
Implement a function which takes a function handle, two $x$ 
values and a limit value as input.
Use the function implemented in exercise
\ref{exSecantMethodOneStep} to improve the root estimates.
Apply the function until the absolute value of the function 
evaluated at the best guess gets below the limit value.
Then return the improved guess, the value of the function 
evaluated at that guess and the number of
iterations used.

The function signature should be:
\begin{lstlisting}
function [x1, fvalue, iterations] = ...
    sekant_method(fh, x0, x1, limit)
\end{lstlisting}
Use the following examples to test the function:
\begin{lstlisting}
>> [x, fval, niter] = sekant_method(...
    @cos, 1.2, 2, 0.01)
x = 1.5724
fval = -0.0016
niter = 1
>> [x, fval, niter] = sekant_method(...
    @sin, 3, 4, 0.01)
x = 3.1395
fval = 0.0021
niter = 2
>> [x, fval, niter] = sekant_method(...
    @sin, 3, 4, 0.001)
x = 3.1416
fval = -7.4395e-08
niter = 3
>> [x, fval, niter] = sekant_method(...
    @(x) log(x) - 1, 3, 4, 0.001)
x = 2.7184
fval = 5.6055e-05
niter = 3
\end{lstlisting}
\begin{hint}
Use a while loop to repeat the calculations as many times 
as needed until the function value gets below the 
specified limit value.
\end{hint}
\begin{sol}
A solution is:
\begin{lstlisting}
function [x1, fvalue, iterations] = sekant_method(fh, x0, x1, limit)

iterations = 0;
fvalue = fh(x1);
while abs(fvalue) > limit
    [x2, fvalue] = sekant_one_step(fh, x0, x1);
    x0 = x1;
    x1 = x2;
    iterations = iterations + 1;
end

end
\end{lstlisting}
\end{sol}
\end{ex}


\begin{ex}
Implement a function that takes two lists
with numerical values as input.
Both lists are sorted, so their values are in 
increasing order.
The two lists should be merged together
in a single list such that the values in 
that list is also sorted.

The function signature should be:
\begin{lstlisting}
function res = merge_ordered_lists(list1, list2)
\end{lstlisting}
Use the following examples to test the function:
\begin{lstlisting}
>> merge_ordered_lists([1, 2, 4], [3, 7, 8])
ans = 1     2     3     4     7     8
>> merge_ordered_lists([1, 2, 4], [])
ans = 1     2     4
>> merge_ordered_lists([], [1, 2, 4])
ans = 1     2     4
\end{lstlisting}
\begin{hint}
Use three while loops.
The first while loop extracts the lowest value from 
the two lists and inserts it in the result.
The loops ends when all values have been used from 
one of the lists.
The second while loop moves all the remaining elements 
in list one to the result.
The third while loop moves all the remaining elements 
in list two to the result.
\end{hint}
\begin{sol}
A solution is:
\begin{lstlisting}
function res = merge_ordered_lists(list1, list2)

res = [];
idx1 = 1;
idx2 = 1;

while idx1 <= length(list1) && idx2 <= length(list2)
    if list1(idx1) < list2(idx2)
        res = [res, list1(idx1)];
        idx1 = idx1 + 1;
    else
        res = [res, list2(idx2)];
        idx2 = idx2 + 1;
    end
end

while idx1 <= length(list1)
    res = [res, list1(idx1)];
    idx1 = idx1 + 1;
end

while idx2 <= length(list2)
    res = [res, list2(idx2)];
    idx2 = idx2 + 1;
end

end
\end{lstlisting}
\end{sol}
\end{ex}



\begin{ex}
Create a function that takes a list of points and a query point
as input.
The function should then determine the point from the 
input list that is nearest the query point.
The index of the nearest point and the distance to the 
query point should then be returned.

The function signature should be:
\begin{lstlisting}
function [min_dist_idx, min_distance] = ...
    locate_nearest_point(input_list, query_point)
\end{lstlisting}
Use the following examples to test the function:
\begin{lstlisting}
>> [idx, dist] = locate_nearest_point([1, 1; 1, 5], [1.1, 1])
idx = 1
dist = 0.1000
>> [idx, dist] = locate_nearest_point([1, 1; 1, 5; 5, 4], [2, 4])
idx = 2
dist = 1.4142
>> [idx, dist] = locate_nearest_point([1, 1; 1, 5], [2, 4])
idx = 2
dist = 1.4142
\end{lstlisting}
\begin{hint}
Use a for loop to iterate over all points in the list of points.
For each point in the list, calculate the distance to the 
reference point and save the index of the point and the
distance to it, if it is the smallest distance seen thus far.
\end{hint}
\begin{sol}
A solution is:
\begin{lstlisting}
function [min_dist_idx, min_distance] = ...
    locate_nearest_point(input_list, query_point)

min_dist_idx = -1;
min_distance = 100000000;

for idx = 1:size(input_list, 1)
    point = input_list(idx, :);
    difference = point - query_point;
    distance = norm(difference);
    if distance < min_distance
        min_dist_idx = idx;
        min_distance = distance;
    end
end

end
\end{lstlisting}
\end{sol}
\end{ex}
 