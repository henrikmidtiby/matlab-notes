% !TeX encoding = UTF-8
% !TeX program = pdfLaTeX
% !TeX root = matlab-exercises-emaip.tex
% !TeX spellcheck = en_GB
\section{Matlab refresher}

\begin{ex}
A store want to implement a 10\% discount to customers 
buying for 600 kr or more.
Create a function that implements the discount calculation
described above, the function must have the signature:
\begin{lstlisting}
function actual_price = apply_discount(ordinary_price)
\end{lstlisting}
Use the following examples to test the function.
\begin{lstlisting}
>> apply_discount(100)
ans = 100
>> apply_discount(600)
ans = 540
>> apply_discount(1000)
ans = 900
>> apply_discount(134)
ans = 134
\end{lstlisting}
\begin{hint}
Use an \verb!if! statement to choose whether the 
discount should be applied or not.
\end{hint}
\begin{sol}
A solution is:
\begin{lstlisting}
function actual_price = apply_discount(ordinary_price)

if(ordinary_price >= 600)
    actual_price = 0.9*ordinary_price;
else
    actual_price = ordinary_price;
end

end
\end{lstlisting}
\end{sol}
\end{ex}




\begin{ex}
\label{exCalcMean}%
Create a function that calculates the mean value $\overline{x}$ of a list of numbers.
\begin{align}
\overline{x}
	& = \frac{1}{N} \sum_{i = 1}^{N} x_i
\end{align}
The function signature should be:
\begin{lstlisting}
function mean_value = calcMean(list)
\end{lstlisting}
Use the following examples to test the function:
\begin{lstlisting}
>> calcMean([1]);
>> calcMean([1])
ans = 1
>> calcMean([1, 3])
ans = 2
>> calcMean([-1, 7])
ans = 3
\end{lstlisting}
\begin{hint}
Use a for loop to sum all values.
\end{hint}
\begin{sol}
A solution is:
\begin{lstlisting}
function mean_value = calcMean(list)

% Initialize counters
total = 0;
n_elements = 0;

% Iterate over values in the list
for value = list
    total = total + value;
    n_elements = n_elements + 1;
end

% Calculate the mean value
mean_value = total / n_elements;

end
\end{lstlisting}
\end{sol}
\end{ex}




\begin{ex}
\label{exMinOfThree}%
Create a function which takes three input arguments
(assumed to be numerical values) and returns the 
minimum value of the three input values.
The function must have the signature:
\begin{lstlisting}
function minval = min_of_three(a, b, c)
\end{lstlisting}
Use the following examples to test the function.
\begin{lstlisting}
>> min_of_three(1, 2, 3)
ans = 1
>> min_of_three(100, 2, 3)
ans = 2
>> min_of_three(-1, 2, -3)
ans = -3
\end{lstlisting}
\begin{hint}
Use two \verb!if! statements to locate the minimal element.
\end{hint}
\begin{sol}
A solution is:
\begin{lstlisting}
function minval = min_of_three(a, b, c)

minval = a;

if(b < minval)
    minval = b;
end

if(c < minval)
    minval = c;
end

end
\end{lstlisting}
\end{sol}
\end{ex}

\begin{ex}
\label{exThreeForThePriceOfTwo}%
A store want to offer a buy three items and pay for two items 
discount (the customer should pay for the two most 
expensive items).
Create a function that takes three input values (the price of the individual items) and returns the total price after taking 
the discount into account.
The function signature should be:
\begin{lstlisting}
function total_price = buy_three_pay_for_two(a, b, c)
\end{lstlisting}
Use the following examples to test the function:
\begin{lstlisting}
>> buy_three_pay_for_two(10, 20, 10)
ans = 30
>> buy_three_pay_for_two(10, 20, 30)
ans = 50
>> buy_three_pay_for_two(30, 20, 35)
ans = 65
\end{lstlisting}
\begin{hint}
The function \verb!min_of_tree! from exercise \ref{exMinOfThree} 
can simplify the calculation.
\end{hint}
\begin{sol}
A solution is:
\begin{lstlisting}
function total_price = buy_three_pay_for_two(a, b, c)

total_price = a + b + c - min_of_three(a, b, c);

end
\end{lstlisting}
\end{sol}
\end{ex}



\begin{ex}
Rewrite the following function such that it uses a for loop.

\begin{lstlisting}
function functionThatRepeatsStuff()
disp(1);
disp(1);
disp(1);
disp(1);
disp(1);
disp(1);
disp(1);
disp(1);
end
\end{lstlisting}
\begin{hint}
Identify what is repeated and how many times it is repeated.
\end{hint}
\begin{sol}
A solution is:
\begin{lstlisting}
function functionThatRepeatsStuff()
for k = 1:8
  disp(1);
end
end
\end{lstlisting}
\end{sol}
\end{ex}



\begin{ex}
Rewrite the following function such that it uses a for loop.
\begin{lstlisting}
function functionThatCounts()
disp(1);
disp(2);
disp(3);
disp(4);
disp(5);
disp(6);
disp(7);
disp(8);
disp(9);
end
\end{lstlisting}
\begin{hint}
Use a for loop, with a structure as shown here:
\begin{lstlisting}
for k = 1:4
end
\end{lstlisting}
\end{hint}
\begin{sol}
A solution is:
\begin{lstlisting}
function functionThatCounts()
for k = 1:9
  disp(k);
end
end
\end{lstlisting}
\end{sol}
\end{ex}



\begin{ex}
Create \label{exSumOfPositiveIntegersLessThanN}
a function that takes one positive integer, n, as argument and
calculates the sum of all integers in the range $1, 2, ..., n$.
The function signature should be:
\begin{lstlisting}
function res = sumOfIntegersLessThanOrEqualTo(n)
\end{lstlisting}
Example usage of the function:
\begin{lstlisting}
>> sumOfIntegersLessThanOrEqualTo(1);
>> sumOfIntegersLessThanOrEqualTo(1)
ans = 1
>> sumOfIntegersLessThanOrEqualTo(5)
ans = 15
>> sumOfIntegersLessThanOrEqualTo(10)
ans = 55
>> sumOfIntegersLessThanOrEqualTo(100)
ans = 5050
\end{lstlisting}
\begin{hint}
Use a for loop, with a structure as shown here:
\begin{lstlisting}
for k = 1:n
end
\end{lstlisting}
Prior to the for loop, create a variable and set its value to 0.
Then update the value in the variable for each cycle in the for loop.
\end{hint}
\begin{sol}
A solution is:
\begin{lstlisting}
function res = sumOfIntegersLessThanOrEqualTo(n)
res = 0;
for k = 1:n
  res = res + k;
end
end
\end{lstlisting}
\end{sol}
\end{ex}




\begin{ex}
Create a function that takes one positive integer, n, as argument and
calculates the sum of all integers in the range $1, 2, ..., n$, which are even.
The function signature should be:
\begin{lstlisting}
function res = sumOfEvenIntegersLessThanOrEqualTo(n)
\end{lstlisting}
Example usage of the function:
\begin{lstlisting}
>> sumOfEvenIntegersLessThanOrEqualTo(1);
>> sumOfEvenIntegersLessThanOrEqualTo(1)
ans = 0
>> sumOfEvenIntegersLessThanOrEqualTo(5)
ans = 6
>> sumOfEvenIntegersLessThanOrEqualTo(10)
ans = 30
>> sumOfEvenIntegersLessThanOrEqualTo(100)
ans = 2550
\end{lstlisting}
\begin{hint}
Modify the solution to exercise \ref{exSumOfPositiveIntegersLessThanN}
so it only increments the variable when $k$ is even.
\end{hint}
\begin{sol}
A solution is:
\begin{lstlisting}
function res = sumOfEvenIntegersLessThanOrEqualTo(n)
res = 0;
for k = 1:n
  if(mod(k, 2) == 0)
    res = res + k;
  end
end
end
\end{lstlisting}
\end{sol}
\end{ex}


\begin{ex}
\label{exCalcSquaredMean}%
Create a function that calculates the mean squared value $\overline{x^2}$ of a list of
numbers.
\begin{align}
\overline{x^2}
	& = \frac{1}{N} \sum_{i = 1}^{N} x_i^2
\end{align}
The function signature should be:
\begin{lstlisting}
function res = calcSquaredMean(list)
\end{lstlisting}
Use the following examples to test the function:
\begin{lstlisting}
>> calcSquaredMean([1]);
>> calcSquaredMean([1])
ans = 1
>> calcSquaredMean([1, 3])
ans = 5
>> calcSquaredMean([-1, 7])
ans = 25
\end{lstlisting}
\begin{hint}
Use a for loop to sum the square of all values.
\end{hint}
\begin{sol}
A solution is:
\begin{lstlisting}
function mean_value = calcSquaredMean(list)

% Initialize counters
total = 0;
n_elements = 0;

% Iterate over values in the list
for value = list
    total = total + value^2;
    n_elements = n_elements + 1;
end

% Calculate the mean value
mean_value = total / n_elements;

end
\end{lstlisting}
\end{sol}
\end{ex}




\begin{ex}
Create a function, that takes a list a input and returns every third element from that list.
The function signature should be:
\begin{lstlisting}
function res = return_every_third_element(list)
\end{lstlisting}
Use the following examples to test the function:
\begin{lstlisting}
>> return_every_third_element([1, 2, 3, 4])
ans = 3
>> return_every_third_element([1])
ans = []
>> return_every_third_element([1, 2, 3, 4, 3, 2])
ans = 3     2
\end{lstlisting}
\begin{hint}
The modulus function \verb!mod! can be of great help
when combined with a \verb!for! loop and an \verb!if! statement.

To add (append) a new element to an existing list, 
use the following code
\begin{lstlisting}
res = []; % Create an empty list.
res = [res, 3]; % Append the value 3 to the list.
\end{lstlisting}
\end{hint}
\begin{sol}
A solution is:
\begin{lstlisting}
function res = return_every_third_element(list)

res = [];
for k = 1:length(list)
    if(mod(k, 3) == 0)
        res = [res, list(k)];
    end
end

end
\end{lstlisting}
\end{sol}
\end{ex}


\begin{ex}
The store from exercise \ref{exThreeForThePriceOfTwo}, would like
to extend the three for the price of two offer, to all items
in the customers shopping basket.
Create a new function that calculates the total price through 
the following steps:
\begin{enumerate}
\item	sort the list of item prices in decreasing order
\item	extract every third element from the list
\item	calculate the sum of all items in the original list (total sum)
\item	calculate the total discount by adding every third element from the sorted list
\item 	calculate the final price by subtracting the sum of thirds from the total sum
\end{enumerate}
The function signature should be:
\begin{lstlisting}
function final_price = three_for_two_offer_many_items(list)
\end{lstlisting}
Use the following examples to test the function:
\begin{lstlisting}
>> three_for_two_offer_many_items([1, 2, 3])
ans = 5
>> three_for_two_offer_many_items([1, 2, 3, 1, 2, 3])
ans = 9
>> three_for_two_offer_many_items([1, 2, 3, 1, 2, 3, 1, 2, 3])
ans = 12
\end{lstlisting}
\begin{hint}
Use the option ``descend'' to the \verb!sort! function.
\end{hint}
\begin{sol}
A solution is:
\begin{lstlisting}
function final_price = three_for_two_offer_many_items(list)

list = sort(list);
final_price = sum(list) - sum(list(1:3:end));

end
\end{lstlisting}
\end{sol}
\end{ex}



\begin{ex}
Implement a function that takes a list as input and returns the mean value $\overline{x}$ and 
standard deviation $\sigma$ of the elements in the list.
Use the following definition of the standard deviation
\begin{align}
\sigma
	& = \sqrt{\overline{x^2} - \overline{x}^2}
\end{align}
Use the functions from exercise \ref{exCalcMean} and
\ref{exCalcSquaredMean}.

The function signature should be:
\begin{lstlisting}
function [m, s] = getMeanAndStandardDeviation(list)
\end{lstlisting}
Use the following examples to test the function:
\begin{lstlisting}
>> getMeanAndStandardDeviation([-1, 7])
ans = 3
>> [m, s] = getMeanAndStandardDeviation([1])
m = 1
s = 0
>> [m, s] = getMeanAndStandardDeviation([-1, 7])
m = 3
s = 4
\end{lstlisting}
\begin{hint}
\end{hint}
\begin{sol}
A solution is:
\begin{lstlisting}
function [m, s] = getMeanAndStandardDeviation(list)

m = calcMean(list);
sqmean = calcSquaredMean(list);
s = sqrt(sqmean - m^2);

end
\end{lstlisting}
\end{sol}
\end{ex}


\begin{ex}
Implement a function that takes a list as input and returns a new list.
The new list should contain two copies of each element in the input list.

The function signature should be:
\begin{lstlisting}
function result = doubleListElements(list)
\end{lstlisting}
Use the following examples to test the function:
\begin{lstlisting}
>> doubleListElements([9]);
>> doubleListElements([1, 2, 3])
ans =  1     1     2     2     3     3
>> doubleListElements([])
ans = []
>> doubleListElements([7.12])
ans = 7.1200    7.1200
\end{lstlisting}
\begin{hint}
Go over all items in the input list.
For each item insert it twice in the output list.
\end{hint}
\begin{sol}
A solution is:
\begin{lstlisting}
function result = doubleListElements(list)

result = []
% Iterate over all values in the list
for value = list
    % Insert each value twice in the result list.
    result = [result, value, value];
end
end
\end{lstlisting}
\end{sol}
\end{ex}


\begin{ex}
Implement a function that takes a list as input and returns a new list.
The new list should only contain half as many elements as the input list, 
elements with even indices should be removed.

The function signature should be:
\begin{lstlisting}
function result = decimateList(list)
\end{lstlisting}
Use the following examples to test the function:
\begin{lstlisting}
>> decimateList([1, 2, 3, 4])
ans =  1     3
>> decimateList([1, 2, 2, 3, 3, 3, 4, 4, 4, 4])
ans = 1     2     3     4     4
>> decimateList([1, 2])
ans =1
>> decimateList([1])
ans = 1
\end{lstlisting}
\begin{hint}
\end{hint}
\begin{sol}
A solution is:
\begin{lstlisting}
function result = decimateList(list)

result = []
% Iterate over list
for idx = 1:length(list)
    % If index is odd
    if(mod(idx, 2) == 1)
        % Insert the element in the result list
        result = [result, list(idx)];
    end
end
end
\end{lstlisting}
\end{sol}
\end{ex}


\begin{ex}
Implement a function that takes a list as input and returns a new list.
The new list should be the input list where the element ordering is reversed.

The function signature should be:
\begin{lstlisting}
function result = reverseList(list)
\end{lstlisting}
Use the following examples to test the function:
\begin{lstlisting}
>> reverseList([1]);
>> reverseList([1])
ans = 1
>> reverseList([1, 2])
ans = 2     1
>> reverseList([1, 2, 3])
ans = 3     2     1
>> reverseList([1, 2, 3, -1])
ans =  -1     3     2     1
>> reverseList([])
ans = []
\end{lstlisting}
\begin{hint}
Go over all elements in the input list.
Insert them in a new list (from the front), one element at
a time.
\end{hint}
\begin{sol}
A solution is:
\begin{lstlisting}
function result = reverseList(list)

result = [];
% Iterate over all values in the list.
for value = list
    % Insert the current value at the front of the result list.
    result = [value, result];
end
end
\end{lstlisting}
\end{sol}
\end{ex}



\begin{ex}
Implement a function that takes a list and a parameter $n$ as 
input and returns a new list.
The new list should be the input list repeated $n$ times.

The function signature should be:
\begin{lstlisting}
function result = repeatList(list, n)
\end{lstlisting}
Use the following examples to test the function:
\begin{lstlisting}
>> repeatList([], 3);
>> repeatList([], 3)
ans = []
>> repeatList([1], 3)
ans =  1     1     1
>> repeatList([1], 0)
ans = []
>> repeatList([1, 2, 3], 3)
ans =  1     2     3     1     2     3     1     2     3
\end{lstlisting}
\begin{hint}
Create an empty list, in which the result should be stored.
Append the input list to the result list and repeat this 
the required number of times.
\end{hint}
\begin{sol}
A solution is:
\begin{lstlisting}
function result = repeatList(list, n)

result = [];
% Repeat n times
for count = 1:n
    % Append the entire input list to the result list.
    result = [result, list];
end
end
\end{lstlisting}
\end{sol}
\end{ex}



\begin{ex}
Implement a function that takes a list, an element to insert into the list and the location of the insertion.
The return value is the list with the new element inserted at the specified position.

The function signature should be:
\begin{lstlisting}
function result = insertElementAtLocation(list, element, position)
\end{lstlisting}
Use the following examples to test the function:
\begin{lstlisting}
>> insertElementAtLocation([1, 2, 3], 6, 2);
>> insertElementAtLocation([1, 2, 3], 6, 6)
Invalid position
>> insertElementAtLocation([1, 2, 3], 6, 2)
ans =  1     6     2     3
>> insertElementAtLocation([], 6, 2)
Invalid position
>> insertElementAtLocation([], 6, 1)
ans =   6
>> insertElementAtLocation([2], 6, 1)
ans = 6     2
>> insertElementAtLocation([2], 6, 2)
ans =  2     6
\end{lstlisting}
\begin{hint}
\end{hint}
\begin{sol}
A solution is:
\begin{lstlisting}
\end{lstlisting}
\end{sol}
\end{ex}


\begin{ex}
Create a function which takes one input argument (a list of numbers)
and returns the last five elements of this list. If the list does not contain five
elements, the function should display a warning.

The function signature should be:
\begin{lstlisting}
function res = lastFiveElements (list)
\end{lstlisting}
Use the following examples to test the function:
\begin{lstlisting}
>> lastFiveElements(1:10);
>> lastFiveElements(1:10)
ans = [6, 7, 8, 9, 10]
>> lastFiveElements(7:10)
warning: there is not enough elements in the list
\end{lstlisting}
\begin{hint}
\end{hint}
\begin{sol}
A solution is:
\begin{lstlisting}
\end{lstlisting}
\end{sol}
\end{ex}


\begin{ex}
Create a function which takes two input argument: a list of numbers
and a single number n and returns the last n elements of list. If list does not
contain n elements, the function should display a warning.

The function signature should be:
\begin{lstlisting}
\end{lstlisting}
Use the following examples to test the function:
\begin{lstlisting}
>> lastNElements(1:10, 3);
>> lastNElements(1:10, 3)
ans = [8, 9, 10]
>> lastNElements(1:10, 10)
ans = [1, 2, 3, 4, 5, 6, 7, 8, 9, 10]
>> lastNElements(7:10, 5)
warning: there is not enough elements in the list
\end{lstlisting}
\begin{hint}
\end{hint}
\begin{sol}
A solution is:
\begin{lstlisting}
\end{lstlisting}
\end{sol}
\end{ex}
 