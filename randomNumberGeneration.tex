% !TeX encoding = UTF-8
% !TeX program = pdfLaTeX
% !TeX root = matlab-exercises-emaip.tex
% !TeX spellcheck = en_GB
\section{Random number generation}

Matlab is able to generate random numbers.
This can be practical in multiple situations.
The random numbers are created by a random number 
generator, which is able to generate numbers that 
(at least) appear to be random, even if a computer 
has generated them.

To simulate a six sided dice, the \verb!randi! command
can be used.
\begin{lstlisting}
>> randi(5)
ans = 3
>> randi(5)
ans = 4
\end{lstlisting}
If a matrix filled with random values are needed, the size
of the matrix can be specified to \verb!randi! as shown here:
\begin{lstlisting}
>> randi(5, 2, 4)
ans = 5     4     3     1
      1     3     1     5
\end{lstlisting}

Similarly can you get a random number from a uniform 
distribution between zero and one by using the \verb!rand! 
function.
\begin{lstlisting}
>> rand()
ans = 0.6385
>> rand()
ans = 0.0336
\end{lstlisting}
In some cases it is beneficial to be able to produce the exact
same sequence of ``random numbers'' multiple times.
This can be achieved by setting the state of the random number 
generator, which is done as follows:
\begin{lstlisting}
>> rng(1); 
>> randi(10, 1, 5)
ans = 5     8     1     4     2
\end{lstlisting}

\subsection{Using random numbers}

One case for using random numbers is to generate 
calculation exercises. 
Below is an example that generates a system of three
equations with three unknowns in the form 
$A \cdot \vec{x} = \vec{b}$.
The system of equations are generated by populating 
$A$ and $\vec{x}$ with random integers.
Then their product is calculated and finally are
$A$ and $\vec{b}$ shown to the user.
\begin{lstlisting}
n = 3;
A = randi([-3, 3], [n, n])
x = randi([-5, 5], [n, 1]);
b = A*x
\end{lstlisting}
The generated output could look like
\begin{lstlisting}
A =  3     3    -3
    -1     3    -2
     1    -3     3
b = -30
     -9
     14
\end{lstlisting}
 