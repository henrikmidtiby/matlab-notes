% !TeX encoding = UTF-8
% !TeX program = pdfLaTeX
% !TeX root = matlab-exercises-emaip.tex
% !TeX spellcheck = en_GB
\section{Linear algebra in Matlab}

Matlab was initially developed as a Matrix Laboratory, ie. a 
place where it was easy to work with matrices and linear algebra.
In this section we will look into how to enter and work with matrices 
and vectors in Matlab.
To enter the matrix 
\begin{align*}
\begin{pmatrix} 1 & 3 & 2 \\ 4 & 8 & 5 \end{pmatrix}
\end{align*}
and save it in the variable \verb!A!, use the following command entered on the 
command line.
\begin{lstlisting}
>> A = [1, 3, 2; 4, 8, 5]
A =
     1     3     2
     4     8     5
\end{lstlisting}
To enter a matrix in Matlab, start with an opening square bracket \verb!"["!.
Then enter all the values in the first row separated by a comma \verb!","!.
To enter the next row, insert a semicolon \verb!";"!.
Fill in values for the next row in a similar way that the values in the first 
row was entered.
Continue until all values in the matrix have been entered and then end
the expression by inserting a closing square bracket \verb!"]"!.
When Matlab shows the value of a matrix, the brackets, commas and 
semicolons are replaced by white space.

When a matrix is saved in a variable, the size / dimensions of the 
matrix can be determined through the \verb!size! command.
To continue with the example from before:
\begin{lstlisting}
>> size(A)
ans =
     2     3
\end{lstlisting}
The output from the size command is a matrix with one row and two columns.
The first value in the output is the number of rows in the matrix, whereas the 
second value in the output is the number of columns in the matrix.

Functions like \verb!ones!, \verb!zeros!, \verb!magic! and \verb!eye! 
all create matrices with a size that is determined by the input 
parameters provided to the functions.


\subsection{Multiplying matrices}

To multiply two matrices you have to use the \verb!*! character.
As an example, lets enter and multiply the following two matrices
\begin{align*}
A = \begin{pmatrix} 4 & 2 \\ -3 & 5 \end{pmatrix} 
\qquad 
B = \begin{pmatrix} 1 & 3 & 2 \\ 4 & 8 & 5 \end{pmatrix} 
\end{align*}
In matlab the operation looks like this:
\begin{lstlisting}
>> A = [4, 2; -3, 5];
>> B = [1, 3, 2; 4, 8, 5];
>> C = A * B
C =
    12    28    18
    17    31    19
\end{lstlisting}
%
If it is not possible to multiply the matrices (when the number of columns in the first 
matrix is different from the number of rows in the second matrix), 
matlab will throw an error like shown here:
\begin{lstlisting}
>> B * A
Error using  * 
Incorrect dimensions for matrix multiplication. Check that the 
number of columns in the first matrix matches the number of rows 
in the second matrix. To perform elementwise multiplication, 
use '.*'.
\end{lstlisting}


\subsection{Solving systems of equations}

It is very easy to solve a system of linear equations with matlab.
As an example take the linear system of two equations with two unknowns
\begin{align*}
1x + 2y = 3 \qquad 3x + 2y = 3
\end{align*}
This linear system of equations can be written in matrix form as follows
\begin{align*}
\begin{pmatrix}
1 & 2 \\ 3 & 2
\end{pmatrix}
\cdot \vec{x} = 
\begin{pmatrix}
3 \\ 3
\end{pmatrix}
\end{align*}
By concatenating the two matrices in the horizontal direction, the 
\emph{augmented} matrix that represents the system of equations can be formed:
\begin{align*}
\begin{pmatrix}
1 & 2 & 3 \\ 3 & 2 & 3
\end{pmatrix}
\end{align*}
Through row operations, the augmented matrix can be converted to reduced 
row echelon form from which it is easy to read out the solution.
This can be achieved with the following code:
\begin{lstlisting}
>> A = [1, 2, 3; 3, 2, 3];
>> rref(A)
ans =
    1.0000         0         0
         0    1.0000    1.5000
\end{lstlisting}

% Slicing, selecting elements from a matrix

% Building matrices from other matrices 
% cat



\todo[inline]{Add exercises about systems of linear equations}
