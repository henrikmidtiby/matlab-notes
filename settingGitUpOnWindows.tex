% !TeX encoding = UTF-8
% !TeX program = pdfLaTeX
% !TeX root = matlab-exercises-emaip.tex
% !TeX spellcheck = en_GB
\section{Setting git up on Windows}

In this section it is explained how to set up git on 
a Windows computer.
This is also described in this \href{https://tekvideo.sdu.dk/t/henrikmidtiby/E-MAIP-2020/2020/1/blok04/4}{video}.

The steps are as follows:
\begin{enumerate}
\item		Download and install \emph{Git for Windows} from \url{https://git-scm.com/}
\item		Open \emph{Git Bash} and identify yourself to git
\item		Generate an ssh key
\item 	Add the generated ssh key to \url{https://gitlab.sdu.dk}
\end{enumerate}

\subsection{Download and installation of \emph{Git for Windows}}

Open \url{https://git-scm.com/} and click on the download button
in the right side of the page.

\subsection{Open Git Bash and identify yourself to git}

Find the \emph{Git Bash} program and start it.
It will look like a black image with some coloured text in the 
top left corner.
What is opened now is a computer console or terminal.
It can be very effective to do certain tasks on the computer through the console.

To set name and email that the git installation uses on the 
computer, the following commands should be executed:
\begin{verbatim}
git config --global user.name "<your name>"
git config --global user.email "<your email>"
\end{verbatim}

For me it looked like this:
\begin{verbatim}
hemi@TEK-CB-HEMI-01 MINGW64 ~
$ git config --global user.name "Henrik Skov Midtiby"

hemi@TEK-CB-HEMI-01 MINGW64 ~
$ git config --global user.email "hemi@mmmi.sdu.dk"
\end{verbatim}

\subsection{Generate an ssh key}

To work efficiently with git on remote servers like \url{https://gitlab.sdu.dk} or \url{https://github.com}, you need to generate 
an ssh key.
In Git Bash it is done with the command.
\begin{verbatim}
ssh-keygen -t rsa
\end{verbatim}
After executing the command, the \emph{ssh-keygen} program
will ask you a few questions and then the ssh key is generated.
It is appropriate to use the default values, so you can just 
press enter to accept the default values.

On my computer the key generation process looked like this:
\begin{verbatim}
hemi@TEK-CB-HEMI-01 MINGW64 ~
$ ssh-keygen -t rsa
Generating public/private rsa key pair.
Enter file in which to save the key (/c/Users/hemi/.ssh/id_rsa):
Enter passphrase (empty for no passphrase):
Enter same passphrase again:
Your identification has been saved in /c/Users/hemi/.ssh/id_rsa
Your public key has been saved in /c/Users/hemi/.ssh/id_rsa.pub
The key fingerprint is:
SHA256:GBdcG4YGzNMGF9bOZ5oK7P+m+IxO1+8hYpT2g3g/+Ok hemi@TEK-CB-HEMI-01
The key's randomart image is:
+---[RSA 3072]----+
|     oo===+      |
|      +o*o.o     |
|      .+.o.      |
|       + .o o    |
|     .. S  =     |
|      o+ +o      |
|     .o.*o= .    |
|     ..Booo= .   |
|     .+o=*Eoo    |
+----[SHA256]-----+
\end{verbatim}

\subsection{Add ssh key to gitlab.sdu.dk}

As you have now generated an ssh key, you need to locate the 
public part of the key and add that to \url{https://gitlab.sdu.dk}.
This can be done with the following command.
\begin{verbatim}
cat <path to public key>
\end{verbatim}

On my computer the key is placed in this location 
\verb!/c/Users/hemi/.ssh/id_rsa.pub!, which was displayed 
in the output from the key generation step.

\begin{verbatim}
hemi@TEK-CB-HEMI-01 MINGW64 ~
$ cat /c/Users/hemi/.ssh/id_rsa.pub
ssh-rsa AAAAB3NzaC1yc2EAAAADAQABAAABgQCz0X6DyHeyiqcPe8Y+zeQ60Y
4F0sxR2MqNlsL52ZrllRd0toAFvPvmrz9/rT13eHoD+xSIQ5kL44GTZqn9fRmC
lG3ttc1v7Qd/KmAmM5eOpr+cH/c/IbfFGQyiRV7lRz5KES8TFUEs8YNLfLwdHU
cEhmn7x4Zmc7aahTQFTH55EqOt6HU5YoF4vMe0LG+WnhNrh/kwiKN1ndqasEC6
wE3D6K2/Z21j8FcFqvR0PWDubUABQdMN5+huQ7/O7SAVQCeZBYoG2fcn9HRu39
IS8kKQ3AmmrDFUitaWhIWZ+msfmTqJCplYXVlGrlvjcbIVPIgCr/4IL6soWE1V
KL1cEVGV2lAktMqP9cU1E6rYMFMtYIIyjSkE5l2YJZ++A+dgayHOqBx5qPXRq2
k2gRAz8auiV0ztOCRtA09IhzvlH0ZHU126ZoxC72BJpMP6+Rp3r8EJ4LlU2IaI
SXnoSAqXtc8HvngXm74TQpneLSDy+nMKZJ1jpL9dqxtWUM3DbhPyzaM= hemi@
TEK-CB-HEMI-01
\end{verbatim}


\subsection{Cloning a git repository}

Open a command line / terminal in the location where you
want to put a local copy of the git repository.
This can be done by right clicking on the folder in 
Pathfinder, Finder or Nautilus and selecting 
\emph{Git BASH here}, \emph{Open terminal} or something similar.
When the terminal is opened in the proper location, issue the 
command below to clone the specified git repository.
\begin{verbatim}
git clone git@gitlab.sdu.dk:midtiby/2020-emaip/
emaip-2020-exercises/emaip-exercises-studentid.git

\end{verbatim}


\subsection{Daily use of git}

The recommended working routine with git is 
described below.
At start of the day, the goal is to get current
with changes from other members of the team.
This is achieved by the commands.
\begin{verbatim}
git fetch
git rebase origin/master
git push
\end{verbatim}
If git complains about not knowing which repository to push
changes to, the following command will
set it up.

\begin{verbatim}
git push --set-upstream origin master
\end{verbatim}

 