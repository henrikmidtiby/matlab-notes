% !TeX encoding = UTF-8
% !TeX program = pdfLaTeX
% !TeX root = matlab-exercises-emaip.tex
% !TeX spellcheck = en_GB
\section{Functions in Matlab}


Write the text for this chapter with the exercises in the folder
\verb!exercises_with_unittests/01_functions! in mind.
This also means that the unit testing framework should be introduced (only introduce how it is used).

Functions are the building blocks of programming languages.
We have already used several of the built in functions 
in Matlab (sqrt, linspace, plot, \ldots). 
By creating your own functions you can extend the functionality of 
Matlab while making it easier to keep an overview of your program.

Functions are especially useful when you need to repeat a calculation 
multiple times.
As an example lets look at how to find the determinant and the two solutions 
to the following quadratic equation: 
\begin{align*}
x^2 + 2x + 4 &= 0 
%x^2 + 4x - 1 &= 0 \\
%4x^2 - 5x + 3 &= 0
\end{align*}
This can be achieved with the Matlab code shown below.
\begin{verbatim}
>> A = 1; 
>> B = 2; 
>> C = -4;
>> D = B^2 - 4 * A * C
D =
    20
>> x1 = (-B - sqrt(D)) / (2*A)
x1 =
   -3.2361
>> x2 = (-B + sqrt(D)) / (2*A)
x2 =
    1.2361
\end{verbatim}



\subsection{Creating functions}


