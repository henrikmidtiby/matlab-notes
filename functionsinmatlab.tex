% !TeX encoding = UTF-8
% !TeX program = pdfLaTeX
% !TeX root = matlab-exercises-emaip.tex
% !TeX spellcheck = en_GB
\section{Functions in Matlab}

Functions are the building blocks of programming languages.
We have already used several of the built in functions 
in Matlab (sqrt, linspace, plot, \ldots). 
By creating your own functions you can extend the functionality of 
Matlab while making it easier to keep an overview of your program.

Functions are especially useful when you need to repeat a calculation 
multiple times.
As an example lets look at how to find the determinant and the two solutions 
to the following quadratic equation: 
\begin{align*}
x^2 + 2x - 4 &= 0 
%x^2 + 4x - 1 &= 0 \\
%4x^2 - 5x + 3 &= 0
\end{align*}
This can be achieved with the Matlab code shown below.
\begin{lstlisting}
>> A = 1; 
>> B = 2; 
>> C = -4;
>> D = B^2 - 4 * A * C
D =
    20
>> x1 = (-B - sqrt(D)) / (2*A)
x1 =
   -3.2361
>> x2 = (-B + sqrt(D)) / (2*A)
x2 =
    1.2361
\end{lstlisting}
This is manageable if we only need to solve a few quadratic equations.
However, if we have to solve quadratic equations, it is much better to 
structure this code in a function, where the values $A$, $B$ and $C$ can be 
specified when calling the function.
To create a function, open a new file, enter the following into the file 
and then save it in the current directory with the name 
\verb!solve_quadratic.m!.
\begin{lstlisting}[numbers=left]
function [sol1, sol2] = solve_quadratic(A, B, C)
D = B^2 - 4 * A * C;
sol1 = (-B - sqrt(D)) / (2*A);
sol2 = (-B + sqrt(D)) / (2*A);
end
\end{lstlisting}
we can now solve the same quadratic as before, with the much shorter code
\begin{lstlisting}
>> [x1, x2] = solve_quadratic(1, 2, -4)
x1 =
   -3.2361
x2 =
    1.2361
\end{lstlisting}

Quite a lot is happening in the code above, so lets look at it 
piece by piece
\begin{lstlisting}[numbers=left]
function [sol1, sol2] = solve_quadratic(A, B, C)
\end{lstlisting}
The first line in the file that contains the function, describe
the fingerprint or \emph{signature} of the function.
The function signature describe what the \emph{name} of the function is, 
which \emph{input} parameters the function accepts and the values that 
the function will \emph{return}.
In this case the name of the function is \verb!solve_quadratic!.
The function takes three input parameters, which are named 
\verb!A!, \verb!B! and \verb!C!.
The function returns the two values \verb!sol1! and \verb!sol2!.
How the return values are calculated is specified in the \emph{body}
of the function, which is the following three lines.
\begin{lstlisting}[numbers=left, firstnumber=2]
D = B^2 - 4 * A * C;
sol1 = (-B - sqrt(D)) / (2*A);
sol2 = (-B + sqrt(D)) / (2*A);
\end{lstlisting}
The last line, that contains the keyword \verb!end! marks the end of the function.
\begin{lstlisting}[numbers=left, firstnumber=5]
end
\end{lstlisting}


\subsection{Exercises about creating functions}

The exercises for this section is placed in the directory 
\verb!01_functions! in the git repository, that you are provided access 
to in class.
The exercises in the git repository are self checking, which means that you 
can get instant feedback on the correctness of the provided solution.
How to work with the exercises, will be shown here. 
As an example lets look at the first exercise.
To do so open the file \verb!exercise_01_test.m! in the Matlab editor.
Then click the \emph{Run Tests} button placed at the top of the screen.
This will generate output similar to what is shown in figure 
\ref{figUnittestExampleOutput}.
Take a closer look at the lines highlighted in yellow.
They tell us that an error in the solution of the exercise has been found.
Line 5 describe in which test the error occurred.
Line 17 to 20 lists the actual return value (0) from the function and the expected value (2).
In Matlab, line 24 appears as a link, which will take you to the location 
where the error was discovered (keep in mind that this is not necessarily 
the location of the error!).


\begin{figure}[p]
\begin{lstlisting}[basicstyle=\scriptsize, breaklines=true, numbers=left, linebackgroundcolor={\lstcolorlines[yellow]{5,17,18,19,20,24,48}}]
>> runtests('exercise_01_test')
Running exercise_01_test

================================================================================
Verification failed in exercise_01_test/test01.
    ---------------------
    Framework Diagnostic:
    ---------------------
    verifyEqual failed.
    --> The numeric values are not equal using "isequaln".
    --> Failure table:
            Actual    Expected    Error    RelativeError
            ______    ________    _____    _____________
        
              0          2         -2           -1      
    
    Actual Value:
         0
    Expected Value:
         2
    ------------------
    Stack Information:
    ------------------
    In /home/hemi/Nextcloud/Work/01_teaching_courses/2021-09-01_EMAIP/materials/notes_on_matlab/exercises_with_unittests/01_functions/exercise_01_test.m (test01) at 12
================================================================================
.
Done exercise_01_test
__________

Failure Summary:

     Name                     Failed  Incomplete  Reason(s)
    ======================================================================
     exercise_01_test/test01    X                 Failed by verification.

ans = 

  TestResult with properties:

          Name: 'exercise_01_test/test01'
        Passed: 0
        Failed: 1
    Incomplete: 0
      Duration: 0.3386
       Details: [1x1 struct]

Totals:
   0 Passed, 1 Failed (rerun), 0 Incomplete.
   0.33863 seconds testing time.
\end{lstlisting}
\caption{Example output from a failed test.}
\label{figUnittestExampleOutput}
\end{figure}


To fix the error, we need to modify the file \verb!return_two.m!, which started with the content
\begin{lstlisting}[numbers=left]
function res = return_two()

% Modify the line below so the function returns the value 2.
res = 0;

end
\end{lstlisting}
The fix consists of changing the 0 in line 4 with a 2; line
4 should now look like:
\begin{lstlisting}[numbers=left, firstnumber = 4]
res = 2;
\end{lstlisting}
To check that the fix, actually solves the problem, lets run the 
test again.
The output of the test run is shown in figure \ref{figExampleOutputFromAPassedTest}.
Line 20 tells us that all test have passed.
This usually means that you can proceed to the next exercise.

\begin{figure}
\begin{lstlisting}[basicstyle=\scriptsize, breaklines=true, numbers=left, linebackgroundcolor={\lstcolorlines[yellow]{20}}]
>> runtests('exercise_01_test')
Running exercise_01_test
.
Done exercise_01_test
__________


ans = 

  TestResult with properties:

          Name: 'exercise_01_test/test01'
        Passed: 1
        Failed: 0
    Incomplete: 0
      Duration: 0.0166
       Details: [1x1 struct]

Totals:
   1 Passed, 0 Failed, 0 Incomplete.
   0.016614 seconds testing time.
\end{lstlisting}
\caption{Example output from a passed test.}
\label{figExampleOutputFromAPassedTest}
\end{figure}

In the \verb!01_functions! directory there is also the file \verb!run_all_tests.m!.
By running the \verb!run_all_tests.m! matlab script, you get a list of all the 
errors that are being found by the set of tests.
A part of the generated output is shown in figure \ref{figRunAllTestsOutput}.
The last part of the \verb!run_all_tests.m! Matlab script will ask you to provide an 
id. Feel free to choose what ever id you want. If the id is not \verb!ukendt!, 
the test results will be sent to a server so that Henrik can monitor the progress 
of the class.

\begin{figure}
\begin{lstlisting}[basicstyle=\scriptsize]
>> run_all_tests

Failure Summary:

     Name                     Failed  Incomplete  Reason(s)
    ======================================================================
     exercise_01_test/test01    X                 Failed by verification.
    ----------------------------------------------------------------------
     exercise_02_test/test01    X         X       Errored.
    ----------------------------------------------------------------------
     exercise_03_test/test01    X                 Failed by verification.
    ----------------------------------------------------------------------
     exercise_03_test/test02    X                 Failed by verification.
    ----------------------------------------------------------------------
     exercise_03_test/test03    X                 Failed by verification.
    ----------------------------------------------------------------------
     exercise_04_test/test01    X         X       Errored.
    ----------------------------------------------------------------------
     exercise_04_test/test02    X         X       Errored.
    ----------------------------------------------------------------------
     exercise_04_test/test03    X         X       Errored.
    ----------------------------------------------------------------------
     exercise_05_test/test01    X                 Failed by verification.
    ----------------------------------------------------------------------
     exercise_05_test/test02    X                 Failed by verification.
    ----------------------------------------------------------------------
     exercise_06_test/test01    X         X       Errored.
    ----------------------------------------------------------------------
     exercise_06_test/test02    X         X       Errored.
    ----------------------------------------------------------------------
     exercise_06_test/test03    X         X       Errored.
    ----------------------------------------------------------------------
     exercise_07_test/test01    X                 Failed by verification.
    ----------------------------------------------------------------------
     exercise_07_test/test02    X                 Failed by verification.
    ----------------------------------------------------------------------
     exercise_07_test/test03    X                 Failed by verification.
    ----------------------------------------------------------------------
     exercise_08_test/test01    X         X       Errored.
    ----------------------------------------------------------------------
     exercise_08_test/test02    X         X       Errored.
    ----------------------------------------------------------------------
     exercise_08_test/test03    X         X       Errored.
    ----------------------------------------------------------------------
     exercise_09_test/test01    X         X       Errored.
    ----------------------------------------------------------------------
     exercise_09_test/test02    X         X       Errored.
    ----------------------------------------------------------------------
     exercise_09_test/test03    X         X       Errored.
Number of passed unittests:       1
Number of failed unittests:      22
Number of incomplete unittests:  13
Sending data to server.
\end{lstlisting}
\caption{Output from run all tests.}
\label{figRunAllTestsOutput}
\end{figure}



\newpage
\subsection{Unplaced exercises}

\begin{ex}
Define the two row vectors $\vec{v}$ and $\vec{u}$:
\begin{align*}
\vec{v} = (1, 2, 3) \qquad \vec{u} = (3, 4, 1)
\end{align*}
and determine their 
sum ($\vec{v} + \vec{u}$),
difference ($\vec{v} - \vec{u}$),
dot product ($\vec{v} \cdot \vec{u}$) and
cross product ($\vec{v} \times \vec{u}$).
In addition compute the elementwise product, division, power of 2 and 
the transposed and length of the two vectors.
\end{ex}
\known{[\ldots], .*, ./, .\^{}2, .', transpose}


\begin{ex}
Define the two matrices
\begin{align*}
\textrm{AMAT} = \begin{pmatrix}
1 &2 &3	\\
2 &4 &6	\\
1 &2 &5
\end{pmatrix}
\qquad
\textrm{BMAT} =
\begin{pmatrix}
1 &0 &0	\\
2 &4 &6	\\
2 &2 &0
\end{pmatrix}
\end{align*}
And calculate their product AMAT $\cdot$ BMAT, element wise product of AMAT times BMAT, sum
AMAT+BMAT, transposed, inverted, determinants, sizes, column and row numbers.
\end{ex}
\known{[,;], \^{}-1, inv, det, size}


\begin{ex}
Use the matrices from the previous exercise. What is the value of \verb!AMAT(5)! and why?
\end{ex}


\begin{ex}
Construct an Identity Matrix $I_{5\times5}$ and a diagonal matrix
\begin{align*}
\textrm{CMAT} = \begin{pmatrix}
2 &0 &0 &0 &0	\\
0 &4 &0 &0 &0	\\
0 &0 &7 &0 &0	\\
0 &0 &0 &0.5 &0	\\
0 &0 &0 &0 &1	
\end{pmatrix}
\end{align*}
And calculate their product $I_{5 \times 5} \cdot \textrm{CMAT}$.
\end{ex}
\known{eye, diag}


\begin{ex}
Solve the set of linear equations
\begin{align*}
1	& = 5x + 2y	\\
0	& = -x + y + 3z + 2v	\\
-10 	& = x + 3y - 4z + v	\\
0	& = -2x + 3z - 2v 
\end{align*}
\end{ex}
\known{$\backslash$}



