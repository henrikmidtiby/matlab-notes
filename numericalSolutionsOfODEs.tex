% !TeX encoding = UTF-8
% !TeX program = pdfLaTeX
% !TeX root = matlab-exercises-emaip.tex
% !TeX spellcheck = en_GB
\section{Numerical solutions of differential equations}
\SetExerciseDirectory{10_numerical_differential_equations}





\subsection{Numerical solutions of differential equations}

Initial value problems, where a differential equation and 
an initial condition are given, can be solved using numeric methods.
Here it is assumed that the differential equation can be 
written in the form
\begin{align}
\label{eqnDiffEquationForm}
\frac{dy}{dx} = f(x, y(x))
\end{align}

\subsubsection{Euler's method}
One of the methods that can be used to approximate the solution
to the initial value problem is Euler's method.
Euler's method is based on the approximation
\begin{align*}
\frac{dy}{dx} = \frac{y(x + h) - y(x)}{h}
\end{align*}
with a step size $h$.
Combined with the differential equation
\eqref{eqnDiffEquationForm}, this lead to the expression
\begin{align*}
\frac{y(x + h) - y(x)}{h} = f(x, y(x))
\end{align*}
by rearranging we obtain
\begin{align*}
y(x + h) = y(x) + h \cdot f(x, y(x))
\end{align*}

Lets look at the initial value problem given below
\begin{align*}
\frac{dy}{dx} = f(x, y(x)) = y	&& y(0) = 1
\end{align*}
To approximate the value of $y(1)$, we can use a single
step with $h = 1$, or multiple steps with a smaller step
length.
First lets examine $h = 1$.
\begin{align*}
y(0) & = 1 \\
y(1) & = y(0) + 1 \cdot f(0, y(0)) \\
& = 1 + 1 \cdot f(0, 1) \\
& = 1 + 1 \cdot 1 \\
& = 2
\end{align*}

Then lets look at $h = 0.5$, now two steps are required to 
estimate the value of $y(1)$.
\begin{align*}
y(0) & = 1 \\
y(0.5) & = y(0) + 0.5 \cdot f(0, y(0)) \\
& = 1 + 0.5 \cdot f(0, 1) \\
& = 1 + 0.5 \cdot 1 \\
& = 1.5 \\
y(1.0) & = y(0.5) + 0.5 \cdot f(0, y(0.5)) \\
& = 1.5 + 0.5 \cdot f(0, 1.5) \\
& = 1 + 0.5 \cdot 1.5 \\
& = 2.25
\end{align*}

When comparing the two approximations for $y(1)$, 
we see that they differ quite much (2 vs. 2.25).
This difference is caused by the approximation error of the method.
The approximation error is reduced when the step length is reduced.


\begin{ex}[Eulers method]%
Implement Euler's method.
The function containing the implementation should
have the signature:
\begin{lstlisting}
function [yvals, fcalls] = eulers_method(fnc, xvals, y0)
% Uses the Euler method for approximating the first order
% differential equation defined by the function handle fnc.
%
% Input values:
% - fnc Function handle to a function which takes two
% input arguments and returns a scalar value.
% Eg. @(x, y) (y+sin(x))
% - xvals List of x values where the corresponding y
% values should be calculated.
% - y0 Initial state of the dependent function.
%
% Output values:
% - yvals Approximation of y(x) at the locations
% specified in xvals.
% - fcalls Number of function evaluations.
\end{lstlisting}
Example usage of the function:
\begin{lstlisting}
>> [yvals, fevals] = eulers_method(@(x, y) y, [0:5], 1)
yvals = 1 2 4 8 16 32
fevals = 5
>> eulers_method(@(x, y) y, [0:5], 2)
ans = 2 4 8 16 32 64
>> eulers_method(@(x, y) 0.1*y, [0:5], 1)
ans = 1.0000 1.1000 1.2100 1.3310 1.4641 1.6105
>> eulers_method(@(x, y) 0.1*y+0.2*x, [0:5], 1)
ans = 1.0000 1.1000 1.4100 1.9510 2.7461 3.8207
\end{lstlisting}
\begin{hint}
No hint provided yet.
\end{hint}
\begin{sol}
No solution provided yet.
\begin{lstlisting}
function [yvals, fevals] = eulers_method(fh, xvals, y0)
%EULERS_METHOD Solves differential equations numerically.

fevals = 0;

% Preallocate array for yvalues (increases speed)
yvals = xvals;

% Set the initial y value
yvals(1) = y0;

% Set current values
curx = xvals(1);
cury = y0;

% Update the current values step by step using Eulers method.
for idx = 2:length(xvals)
    newx = xvals(idx);
    derivative = fh(curx, cury);
    fevals = fevals + 1;
    dx = newx - curx;
    dy = derivative * dx;
    curx = newx;
    cury = cury + dy;
    yvals(idx) = cury;
end

end
\end{lstlisting}
\end{sol}
\end{ex}

\begin{ex}
\label{excEulerConvergence} \\
For the initial value problem
\begin{align}
y'(t)
	& = 1.2 \cdot y(t) 	&
y(0) = 1
\end{align}
Determine the analytic solution and calculate the exact value of $y(3)$.
Use Euler's method to approximate $y(3)$ with different step lengths $h$.
How is the error related to the used step length?
\begin{hint}
No hint provided yet.
\end{hint}
\begin{sol}
No solution provided yet.
\end{sol}
\end{ex}


\subsection{Heun's method}

Euler's method is rather simple to implement, however the 
error in its approximations are quite large for a certain step 
length.
There exist methods with a much lower error and here we will look 
at one of them, Heun's method.
The method is based on the following rules:
\begin{align*}
\tilde{y}(x + h) & = y(x) + h \cdot f(x, y(x)) \\
y(x + h) & = y(x) + \frac{h}{2} \left[f(x, y(x)) + f(x + h, \tilde{y}(x + h))\right]
\end{align*}




\begin{ex}[Heun's method]%
Implement Heuns's method.
The function containing the implementation should
have the signature:
\begin{lstlisting}
function [yvals, fcalls] = heuns_method(fnc, xvals, y0)
% Uses Heun's method for approximating the first order
% differential equation defined by the function handle fnc.
%
% Input values:
% - fnc Function handle to a function which takes two
% input arguments and returns a scalar value.
% Eg. @(x, y) (y+sin(x))
% - xvals List of x values where the corresponding y
% values should be calculated.
% - y0 Initial state of the dependent function.
%
% Output values:
% - yvals Approximation of y(x) at the locations
% specified in xvals.
% - fcalls Number of function evaluations.
\end{lstlisting}
Example usage of the function:
\begin{lstlisting}
>> [yvals, fevals] = heuns_method(@(x, y) y, [0:5], 1)
yvals = 1.0000    2.5000    6.2500   15.6250   39.0625   97.6562
fevals = 10
>> heuns_method(@(x, y) y, [0:5], 2)
ans = 2.0000    5.0000   12.5000   31.2500   78.1250  195.3125
>> heuns_method(@(x, y) 0.1*y, [0:5], 1)
ans = 1.0000    1.1050    1.2210    1.3492    1.4909    1.6474
>> heuns_method(@(x, y) 0.1*y+0.2*x, [0:5], 1)
ans = 1.0000    1.2050    1.6415    2.3339    3.3089    4.5964
\end{lstlisting}
\begin{hint}
No hint provided yet.
\end{hint}
\begin{sol}
No solution provided yet.
\begin{lstlisting}
function [yvals, fevals] = eulers_method(fh, xvals, y0)
%EULERS_METHOD Solves differential equations numerically.

fevals = 0;

% Preallocate array for yvalues (increases speed)
yvals = xvals;

% Set the initial y value
yvals(1) = y0;

% Set current values
curx = xvals(1);
cury = y0;

% Update the current values step by step using Eulers method.
for idx = 2:length(xvals)
    newx = xvals(idx);
    derivative = fh(curx, cury);
    fevals = fevals + 1;
    dx = newx - curx;
    dy = derivative * dx;
    curx = newx;
    cury = cury + dy;
    yvals(idx) = cury;
end

end
\end{lstlisting}
\end{sol}
\end{ex}
 