% !TeX encoding = UTF-8
% !TeX program = pdfLaTeX
% !TeX root = matlab-exercises-emaip.tex
% !TeX spellcheck = en_GB
\section{Complex numbers}

\begin{ex}
Check some of your calculations on the exercise sheet about complex
numbers using matlab.
\begin{hint}
\end{hint}
\begin{sol}
A solution is:
\begin{lstlisting}
\end{lstlisting}
\end{sol}
\end{ex}

\begin{ex}
Plot the two complex numbers $a = 4 + i$ and 
$b = 2 \cdot e^{i \pi / 3}$ and their
product in the same Argand diagram.
\begin{hint}
\end{hint}
\begin{sol}
A solution is:
\begin{lstlisting}
% Enter values.
a = 4 + 1*i;
b = 2*exp(i*pi/3);

% Plot values
figure(1);
values = [0, a, b, a*b];
plot(real(values), imag(values), 'o');
axis equal;
\end{lstlisting}
\end{sol}
\end{ex}


\begin{ex}
Copy the code below into a file and try to run it a few times.
\begin{lstlisting}
figure(1);
clf
hold off;
extend = 4;
xlim([-extend, extend]);
ylim([-extend, extend]);
hold on;
plot([-extend, extend], [0, 0], '-k');
plot([0, 0], [-extend, extend], '-k');
z1 = ginput(1) * [1; 1i];
plot([0, real(z1)], [0, imag(z1)], '.-r', 'MarkerSize', 10);
z2 = ginput(1) * [1; 1i];
plot([0, real(z2)], [0, imag(z2)], '.-r', 'MarkerSize', 10);
z3 = z1 * z2;
plot([0, real(z3)], [0, imag(z3)], '-b', 'MarkerSize', 10);\end{lstlisting}
\begin{hint}
\end{hint}
\begin{sol}
A solution is:
\begin{lstlisting}
\end{lstlisting}
\end{sol}
\end{ex}



\begin{ex}
The function signature should be:
\begin{lstlisting}
%% 
% Write complex numbers in matlab
a = 3 + 4*i
b = 2 * exp(i*1)


%%
% Standard operations on complex numbers
a + b
a - b
a * b
a / b


%%
% Conversion of complex numbers to and from polar form.

% Determine modulus and argument.
abs(a)
angle(a)

% Determine real and imaginary parts.
real(b)
imag(b)




\end{lstlisting}
\begin{hint}
\end{hint}
\begin{sol}
A solution is:
\begin{lstlisting}
\end{lstlisting}
\end{sol}
\end{ex}

\subsection{Functions}

\begin{ex}
\begin{lstlisting}
%%
% Testing the square root function.
% Can you improve the function, so the result is more accurate?
x = square_root(5)
x*x

x = square_root(10)
x*x

x = square_root(50)
x*x


%%
a = 4;
if(a > 3)
    disp('Hello');
end

%%
k = 4;
x = 0;
while(x < k)
    disp(x);
    x = x + 1;
end

%%
x = input('Specify a positive integer: ');
k = 2;
while(k < x)
    if(mod(x, k) == 0)
        disp('The number is not prime, a divisor is: ');
        disp(k);
        %break;
    end
    k = k + 1;
end
if(k == x)
    disp('The number is a prime');
end



mod(6, 4)
\end{lstlisting}
Use the following examples to test the function:
\begin{lstlisting}
\end{lstlisting}
\begin{hint}
\end{hint}
\begin{sol}
A solution is:
\begin{lstlisting}
\end{lstlisting}
\end{sol}
\end{ex}


 