% !TeX encoding = UTF-8
% !TeX program = pdfLaTeX
% !TeX root = matlab-exercises-emaip.tex
% !TeX spellcheck = en_GB
\section{Numerical integration}
\SetExerciseDirectory{08_numerical_integration}

\subsection{Trapezoidal rule}
It is not always possible to evaluate the definite integral
\[
\int_a^b f(x) \, dx
\]
as it can be difficult or even impossible to determine the 
anti--derivative $F(x)$.

The value of the definite integral can however be estimated
using numerical methods.
The Trapezoidal rule is one approach for estimating the 
numerical value of the integral, the Trapezoidal rule 
is based on the approximation:
\[
\int_a^b f(x) \, dx \simeq (b - a) \cdot \frac{f(a) + f(b)}{2}
\]

The error of the approximation depends on the function that 
is being integrated $f(x)$ and the length of the integration 
interval.
To reduce the error, the range of the integral can be divided 
into two and then the approximation is applied to each interval.
\begin{align*}
\int_a^b f(x) \, dx 
& = \int_a^c f(x) \, dx + \int_c^b f(x) \, dx \\
& \simeq (c - a) \cdot \frac{f(a) + f(c)}{2} + (b - c) \cdot \frac{f(c) + f(b)}{2}
\end{align*}


\begin{ex}
Explain how the Trapezoidal rule works.
\[
\int_a^b f(x) \, dx \simeq (b - a) \cdot \frac{f(a) + f(b)}{2}
\]
\begin{hint}
Draw a function and the area below the function over the r
range of integration.
Can you recognize the values $(b - a)$ and $\frac{f(a) + f(b)}{2}$
in the figure?
\end{hint}
\begin{sol}
No solution provided yet.
\end{sol}
\end{ex}

\begin{ex}
Plot the function 
\[
f(x) = e^{-x} \cdot \sin(x)
\]
over the range $x \in [0, 5]$.
\begin{hint}
Define a function handle.
Make a list of x values that covers the range.
Apply the function handle to the x values.
Plot the calculated function values against the x values.
\end{hint}
\begin{sol}
A solution is:
\begin{lstlisting}
figure(1);
f = @(x) exp(-x) .* sin(x);
x = linspace(0, 5);
plot(x, f(x));
\end{lstlisting}
\end{sol}
\end{ex}


\begin{ex}[Reference value]\label{exTrapezRuleReference}%
Evaluate the integral below numerically using the \verb!integral! function.
\[
\int_0^5 e^{-x} \cdot \sin(x) \, dx
\]
\begin{hint}
The two first significant digits are 0.50.
\end{hint}
\begin{sol}
A solution is:
\begin{verbatim}
fh = @(x) exp(-x) .* sin(x);
integral(fh, 0, 5)
\end{verbatim}
\end{sol}
\end{ex}

\begin{ex}[Trapezoidal rule example]\label{exTrapezRule}%
Create a list of 10 evenly spaced $x$--values from 
0 to 5, these values will be called $x_i$.
Evaluate the function 
\[
f(x) = e^{-x} \cdot \sin(x)
\]
at these $x$--values and save these values in 
a list called $y_i$.
Then calculate the sum
\[
\sum_{i = 1}^{9} \frac{y_i + y_{i + 1}}{2} \cdot (x_{i + 1} - x_i)
\]
\begin{hint}
Use \verb!linspace! to generate a list of x-values.
\end{hint}
\begin{sol}
A solution based on a for loop is
\begin{lstlisting}
nvals = 10;
x = linspace(0, 5, nvals);

fh = @(x) exp(-x) .* sin(x);
fx = fh(x);

value = 0;
for k = 1:(nvals - 1)
    avg_height = 0.5*(fx(k) + fx(k + 1));
    dx = x(k + 1) - x(k);
    value = value + avg_height * dx;
end
\end{lstlisting}

A more compact solution based on vectorization
is 
\begin{lstlisting}
nvals = 10;
x = linspace(0, 5, nvals);

fh = @(x) exp(-x) .* sin(x);
fx = fh(x);

dx = x(2:end) - x(1:end-1);
avg = 0.5 * (fx(1:end-1) + fx(2:end));
value = dx * avg';
\end{lstlisting}
\end{sol}
\end{ex}

\begin{ex}[Error]%
Calculate the difference between the approximation from 
exercise \ref{exTrapezRule} and the reference values from 
exercise \ref{exTrapezRuleReference}.
\begin{hint}
\end{hint}
\begin{sol}
A solution is:
\begin{lstlisting}
\end{lstlisting}
\end{sol}
\end{ex}


\begin{ex}[Reduced step length]%
Repeat exercise \ref{exTrapezRule}, but use 20 steps instead of 10.
\begin{hint}
Look at the solution to \ref{exTrapezRule} and modify the call to linspace.
\end{hint}
\begin{sol}
\begin{lstlisting}
nvals = 20;
x = linspace(0, 5, nvals);

fh = @(x) exp(-x) .* sin(x);
fx = fh(x);

value = 0;
for k = 1:(nvals - 1)
    avg_height = 0.5*(fx(k) + fx(k + 1));
    dx = x(k + 1) - x(k);
    value = value + avg_height * dx;
end
\end{lstlisting}
\end{sol}
\end{ex}

\begin{ex}[Trapezoidal rule as a function]%
Create a function that uses the Trapezoidal rule to estimate the 
value of a definite integral.
The function must have the signature
\begin{lstlisting}
function res = trapez_rule(fh, x_low, x_high, n_intervals)
\end{lstlisting}
where \verb!fh! is a function handle, 
\verb!x_low! and \verb!x_high! are the integration limits
and \verb!n_intervals! is the number of subintervals 
used for evaluating the integral.
Use the following examples to test the function
\begin{lstlisting}
>> fh = @(x) exp(-x) .* sin(x);
>> trapez_rule(fh, 0, 5, 10)
ans = 0.4770
>> trapez_rule(fh, 0, 5, 20)
ans = 0.4966
>> trapez_rule(fh, 0, 5, 100)
ans = 0.5021
\end{lstlisting}
\begin{hint}
Use your solution from exercise \ref{exTrapezRule} and build a function around it.
\end{hint}
\begin{sol}
A solution is:
\begin{lstlisting}
function res = trapez_rule(fh, x_low, x_high, n_intervals)

x = linspace(x_low, x_high, n_intervals);
fx = fh(x);
res = 0;
for k = 1:(n_intervals - 1)
    avg_height = 0.5*(fx(k) + fx(k + 1));
    dx = x(k + 1) - x(k);
    res = res + avg_height * dx;
end

end
\end{lstlisting}
\end{sol}
\begin{solutionfile}{trapez_rule.m}
function res = trapez_rule(fh, x_low, x_high, n_intervals)

x = linspace(x_low, x_high, n_intervals);
fx = fh(x);
res = 0;
for k = 1:(n_intervals - 1)
    avg_height = 0.5*(fx(k) + fx(k + 1));
    dx = x(k + 1) - x(k);
    res = res + avg_height * dx;
end

end
\end{solutionfile}
\begin{solutionfile}{trapez_rule_test.m}
function tests = trapez_rule_test
    tests = functiontests(localfunctions);
end

%% Test 1: Uniform sampling
function test1(testCase)
    fh = @(x) exp(-x) .* sin(x);
    actual_value =  trapez_rule(fh, 0, 5, 10);
    expected_value = 0.4770;
    testCase.verifyEqual(actual_value, expected_value, ...
        'RelTol', 0.001);
end

%% Test 2: 
function test2(testCase)
    fh = @(x) exp(-x) .* sin(x);
    actual_value =  trapez_rule(fh, 0, 5, 20);
    expected_value = 0.4966;
    testCase.verifyEqual(actual_value, expected_value, ...
        'RelTol', 0.001);
end

\end{solutionfile}
\end{ex}

\begin{ex}
Investigate how the numerical error of the estimate decreases 
as the number of intervals are increased.
Make a table of the used step lengths and the 
associated errors.
To do this you will have to select a function and an 
interval over which the definite integral should be
calculated.
\begin{hint}
Call the \verb!trapez_rule! function with different 
number of steps.
For each call of the function, write the step length
and the error to the console.
\end{hint}
\begin{sol}
A solution is: 
\begin{lstlisting}
fh = @(x) exp(-x) .* sin(x);
xlow = 0;
xhigh = 5;
ref = integral(fh, xlow, xhigh);

for k = 1:10
    steps = 2^k;
    val = trapez_rule(fh, xlow, xhigh, steps);
    fprintf("%8.3f %8.3f\n", (xhigh - xlow) / steps, val - ref);
end
\end{lstlisting}
The error is seen to be proportional to the step length
squared.
\end{sol}
\end{ex}


\begin{ex}
Estimate the values of the integrals
\[
V_1 = \int_1^3 \sin(2x) \, dx \qquad \textrm{and} \qquad
V_2 = \int_2^6 \frac{\sin(x)}{2} \, dx.
\]
How would you expect them to be related?
\begin{hint}
Use the \verb!integral! function.
The substitution rule has been applied to the first
integral and this has generated the second integral.
\end{hint}
\begin{sol}
A solution is
\begin{lstlisting}
fh1 = @(x) sin(2*x);
fh2 = @(x) sin(x) / 2;
integral(fh1, 1, 3)
integral(fh2, 2, 6)
\end{lstlisting}

As the integrals are related to each other 
through the substitution rule, they should have 
the exact same value; which is also confirmed 
using matlab.
\end{sol}
\end{ex}



 