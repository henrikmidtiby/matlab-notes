% !TeX encoding = UTF-8
% !TeX program = pdfLaTeX
% !TeX root = matlab-exercises-emaip.tex
% !TeX spellcheck = en_GB
\section{Numerical root finding}

\todo[inline]{Add exercises about plotting phase space diagrams for differential equations}
\todo[inline]{Examine the stability about equilibria in differential equations}

In the next set of exercises the goal will be to find solutions to the equation
%
\begin{align}
x
	& = \exp(x - 2)
\end{align}
%
The first step is to rewrite the equation such that one of the sides become 0.
%
\begin{align}
0
	& = \exp(x - 2) - x
\end{align}
%
The right hand side will be denoted $f(x)$ and the problem is now to find the 
roots of the function $f(x)$.
\begin{align}
0 & = f(x)
\end{align}

\begin{ex}
Locate the two roots of the function graphically using the plot command in matlab.
To ensure precise readings use the zoom functionality.
Give the roots with at least three correct decimals.
\end{ex}


\subsection{Fixed point iteration}

\begin{align}
f_0(x_0)
	& = f(x_0)	\\
f_n(x_0)
	& = f_{n - 1}(x_0)
\end{align}


\begin{ex}
Create a function which performs fixed point iterations on a given
function.
The function should take three input arguments, a function handle $f$, the initial x value 
($x_0$) and the number of iterations ($n$).
The return value should be $f_n(x_0)$.
The intermediate calculations $f_0(x_0)$, $f_1(x_0)$, $f_2(x_0)$, \ldots 
should be displayed on screen.

\begin{lstlisting}
function res = fixedIter(x0, n)
\end{lstlisting}
Example usage of the function:
\begin{lstlisting}
>> fnc = @(x) exp(x - 2);
>> fixedIter(fnc, 1, 2)
    0.3679
    0.1955
    0.1646
ans =
    0.1646
\end{lstlisting}
\end{ex}

\begin{ex}
Does the function fixedIter always converge to the same value when used on the function
handle given in the exercise above?
If the function converges, how is the converged value related 
to the equation $x = \exp(x - 2)$?
\end{ex}




\begin{ex}
Implement a function which uses bisection to locate roots in a given function. 
As input should it take a function handle, the initial interval (given as 
lower and higher values) and the number of iterations.
If the function have opposite signs at the lower and higher values the 
bisection method should run, in the other case a warning should be issued.
As output the function should return an array containing four columns and a number of 
rows equal to the number of iterations.
Each row should have the values \emph{lower}, \emph{higher}, \emph{fvalLower} and 
\emph{fvalHigher} (the current limits and the function values at the limits).
 
\begin{lstlisting}
function data = bisection(fnc, lower, higher, iterations)
\end{lstlisting}
Example usage of the function:
\begin{lstlisting}
>> fnc = @(x) exp(x - 2) - x;
>> bisection(fnc, 0, 0.1, 5)
Bad starting values
ans =[]
>> bisection(fnc, 0, 1, 5)
ans =
         0    0.5000    0.1353   -0.2769
         0    0.2500    0.1353   -0.0762
    0.1250    0.2500    0.0284   -0.0762
    0.1250    0.1875    0.0284   -0.0243
    0.1563    0.1875    0.0020   -0.0243
\end{lstlisting}
\end{ex}


\begin{ex}\\
Investigate how the bisection method converges. 
E. g. how many iterations is required to reach a given accuracy?
\end{ex}

\begin{ex}
Implement a function which uses the secant method to locate roots in a given function.
As input should it take a function handle and two different $x$ values near the root 
that should be located and the number of iterations.
If the method converges before the requested number of iterations it should halt 
(the method has converged when the calculated function value is zero).
After each iteration the method should output the new estimate of the root location $x$
and the function value at $x$.
The secant method is described here \url{http://en.wikipedia.org/wiki/Secant_method}.

\begin{lstlisting}
function secant(fnc, x0, x1, iterations)
\end{lstlisting}
Example usage of the function:
\begin{lstlisting}
>> fnc = @(x) exp(x - 2) - x;
>> format long
>> secant(fnc, 3, 3.4, 3)
   3.400000000000000   0.655199966844675
   3.120274401785570  -0.054579081629965
   3.141784146672724  -0.009432188618255
>> secant(fnc, 0.1, 0.2, 9)
   0.200000000000000  -0.034701111778413
   0.158821380623629  -0.000191029545640
   0.158593437586678   0.000000758928080
   0.158594339582341  -0.000000000016240
   0.158594339563039                   0
\end{lstlisting}
\end{ex}



\begin{ex}\\
Implement a function which uses the regula falsi method locate roots in a given function.
As input should it take a function handle, the initial interval (given as 
lower and higher values) and the number of iterations.
If the function has opposite signs at the lower and higher values the 
regula falsi method should run, in the other case a warning should be issued.
As output the function should return an array containing four columns and a number of 
rows equal to the number of iterations.
Each row should have the values \emph{lower}, \emph{higher}, \emph{fvalLower} and 
\emph{fvalHigher} (the current limits and the function values at the limits).

\begin{lstlisting}
function regulafalsi(fnc, lower, higher, iterations)
\end{lstlisting}
Example usage of the function:
\begin{lstlisting}
>> fnc = @(x) exp(x - 2) - x;
>> regulafalsi (fnc, 0 ,  -1,  5)
Bad starting values
ans = []
>> regulafalsi (fnc, 0 ,  0.1,  5)
Bad starting values
ans = []
>> regulafalsi (fnc, 0 ,  1,  4)
ans =
         0    0.1763    0.1353   -0.0149
         0    0.1588    0.1353   -0.0002
         0    0.1586    0.1353   -0.0000
         0    0.1586    0.1353   -0.0000
>> result = regulafalsi (fnc, -10 ,  1,  5)
result =
  -10.0000    0.3460   10.0000   -0.1547
  -10.0000    0.1884   10.0000   -0.0250
  -10.0000    0.1630   10.0000   -0.0037
  -10.0000    0.1592   10.0000   -0.0005
  -10.0000    0.1587   10.0000   -0.0001
\end{lstlisting}
\end{ex}

\begin{ex}\\
Is the regula falsi method working well in the example given above? 
Try to compare with the convergence of the bisection method, which converges 
most quickly?
Try to suggest a combination of the two methods that will converge faster than 
either bisection and regula falsi.
\end{ex}



\begin{ex}\\
Create a function which implements Newtons method. 
As input should it take two function handles (one for $f(x)$ and one for $f'(x)$), 
an initial guess $x_0$ and the number of iterations.
As output should it return a list of the calculated $x$ values, one value for each performed iteration.
\begin{lstlisting}
function result = newton(fnc, dfnc, x0, iterations )
\end{lstlisting}
Example usage of the function:
\begin{lstlisting}
>> fnc = @(x) cos(x);
>> dfnc = @(x) - sin(x);
>> format long;
>> vals = newton(fnc, dfnc, 1, 2)
vals = 1.000000000000000   1.642092615934331   1.570675277161251
\end{lstlisting}
\end{ex}




\begin{ex}\\
Compare the convergence properties of the implemented functions for rootfinding, 
ie how many iterations are required to reach a function value below $10^{-1}$,
$10^{-2}$, $10^{-4}$, $10^{-8}$, $10^{-12}$\,?

Use the following equation as case: $f(x) = \exp(x - 4) + \cos(x)$.
As initial values use $x_0 = 5$ for Newtons method and the ends of the range 
$x \in [2; 5]$ for the methods requiring two initial values.
Make table like the one below, which contains the number of iterations used for 
each method to find a function value less than a given threshold.

\begin{centering}
\begin{tabular}{l|r|r|r|r|r}
Method \textbackslash Function value	& $10^{-1}$ & $10^{-2}$ & $10^{-4}$ & $10^{-8}$ & $10^{-12}$ \\
\hline
Newton	&  	& 3 	&  	&  	&	\\
Secant	& 4	&	& 	& 	&	\\
Bisection	&	&	&	&	&	\\
Regula falsi &	&	&	& 	& 36	\\
\end{tabular}
\end{centering}
\end{ex}
 